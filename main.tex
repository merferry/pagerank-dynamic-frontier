\documentclass[sigconf,nonacm]{acmart}

%% Enable subfigures
\usepackage{subfigure}
%% Enable numbers in scientific format.
\usepackage{siunitx}
%% Enable enumerate start from.
\usepackage{enumitem}

%% Enable theorems
\usepackage{amsmath}
\newtheorem{theorem}{Theorem}[section]
\newtheorem{lemma}[theorem]{Lemma}

%% Enable algorithms
\usepackage{algorithm}
\usepackage[noend]{algpseudocode}
\let\ReturnInline\Return
\renewcommand{\Return}{\State\ReturnInline}
\algrenewcommand\algorithmicrequire{$\rhd$}
\algrenewcommand\algorithmicensure{$\square$}

%% Fonts used in the template cannot be substituted; margin 
%% adjustments are not allowed.
\AtBeginDocument{%
  \providecommand\BibTeX{{%
    \normalfont B\kern-0.5em{\scshape i\kern-0.25em b}\kern-0.8em\TeX}}}

%% Rights management information.
\setcopyright{acmcopyright}
\copyrightyear{2018}
\acmYear{2018}
\acmDOI{XXXXXXX.XXXXXXX}

%% These commands are for a PROCEEDINGS abstract or paper.
\acmConference[Conference acronym 'XX]{Make sure to enter the correct
  conference title from your rights confirmation emai}{June 03--05,
  2018}{Woodstock, NY}
%% Title of the proceedings is different from ``Proceedings of ...''?
% \acmBooktitle{Woodstock '18: ACM Symposium on Neural Gaze Detection,
%  June 03--05, 2018, Woodstock, NY} 
% \acmPrice{15.00}
% \acmISBN{978-1-4503-XXXX-X/18/06}

%% Submission ID.
% \acmSubmissionID{123-A56-BU3}

%% Use the "author year" style of citations and references?
% \citestyle{acmauthoryear}

%% Message
\newcommand{\kk}[1]{{{\color{red} #1}}}
\newcommand{\ds}[1]{{{\color{blue} #1}}}
\newcommand{\su}[1]{{{\color{green} #1}}}

%% Ignore block
\newcommand{\ignore}[1]{}




\begin{document}

%% Full title of the paper.
\title[DF* PageRank: Improved Incrementally Expanding Approaches for Updating PageRank on Dynamic Graphs]{DF* PageRank: Improved Incrementally Expanding Approaches for Updating PageRank on Dynamic Graphs}

%% Short title to be used in page headers (optional).
% \title[short title]{full title}
% \subtitle{Something other than the title}

%% Authors and their affiliations.
\author{Subhajit Sahu}
\email{subhajit.sahu@research.iiit.ac.in}
\affiliation{%
  \institution{IIIT Hyderabad}
  \streetaddress{Professor CR Rao Rd, Gachibowli}
  \city{Hyderabad}
  \state{Telangana}
  \country{India}
  \postcode{500032}
}

%% Concise author list in page headers.
%\renewcommand{\shortauthors}{Sahu, Kothapalli, and Banerjee, et al.}

%% Show page numbers.
\settopmatter{printfolios=true}

%% Short summary of the work to be presented in the article.
\begin{abstract}
PageRank is a widely used centrality measure that assesses the significance of vertices in a graph by considering their connections and the importance of those connections. Efficiently updating PageRank on dynamic graphs is essential for various applications due to the increasing scale of datasets. This technical report introduces our improved Dynamic Frontier (DF) and Dynamic Frontier with Pruning (DF-P) approaches. Given a batch update comprising edge insertions and deletions, these approaches iteratively identify vertices likely to change their ranks with minimal overhead. On a server featuring a 64-core AMD EPYC-7742 processor, our approaches outperform Static and Dynamic Traversal PageRank by $5.2\times$/$15.2\times$ and $1.3\times$/$3.5\times$ respectively - on real-world dynamic graphs, and by $7.2\times$/$9.6\times$ and $4.0\times$/$5.6\times$ on large static graphs with random batch updates. Furthermore, our approaches improve performance at a rate of $1.8\times$/$1.7\times$ for every doubling of threads.
\end{abstract}




%% The code below is generated by the tool at http://dl.acm.org/ccs.cfm.
\begin{CCSXML}
<ccs2012>
<concept>
<concept_id>10003752.10003809.10010170</concept_id>
<concept_desc>Theory of computation~Parallel algorithms</concept_desc>
<concept_significance>500</concept_significance>
</concept>
<concept>
<concept_id>10003752.10003809.10003635</concept_id>
<concept_desc>Theory of computation~Graph algorithms analysis</concept_desc>
<concept_significance>500</concept_significance>
</concept>
</ccs2012>
\end{CCSXML}

% \ccsdesc[500]{Theory of computation~Parallel algorithms}
% \ccsdesc[500]{Theory of computation~Graph algorithms analysis}

%% Pick words that accurately describe the work being presented.
\keywords{Parallel PageRank algorithm, Improved Dynamic Frontier approach}

% \received{20 February 2007}
% \received[revised]{12 March 2009}
% \received[accepted]{5 June 2009}




%% Process the author and title information.
\maketitle

\section{Introduction}
\label{sec:introduction}
Centrality is the problem of quantifying the significance of nodes within a network based on link structures. PageRank \cite{rank-page99}, originally proposed to rank web pages in search results, is one the most popular centrality metrics. It is based on the principle that pages receiving a greater number of high-quality links are of higher quality and, consequently, should be assigned higher ranks. Due to the importance of such a metric, its also finds applications in areas other than web page ranking, such as, urban planning \cite{urban-zhang18}, traffic flow prediction \cite{traffic-kim15}, and protein target identification \cite{banky2013equal}.

The escalating availability of graph-based data has spurred a considerable interest in parallel algorithms for PageRank computation \cite{rank-garg16, rank-nvgraph, rank-giri20, rank-sarma13}.\ignore{--- implementations span multicore CPUs \cite{rank-garg16}, GPUs \cite{rank-nvgraph}, FPGAs \cite{rank-guoqiang20}, SpMV ASICs \cite{rank-sadi18}, CPU-GPU hybrids \cite{rank-giri20}, CPU-FPGA hybrids \cite{rank-li21}, and distributed systems \cite{rank-sarma13}.}

%% FROM COMPRE
PageRank is used for word sense \textbf{disambiguation} in lexical semantics (PPR) \cite{de2010robust, duque2015co, agirre2010graph}, urban planning \cite{urban-zhang18}, ranking streets by traffic \cite{traffic-kim15}, identifying \textbf{communities} \cite{rank-kloumann17}, measuring their \textbf{impact} on the web, maximizing influence \cite{influence-zhang15}, providing \textbf{recommendations} (PPR) \cite{recommend-chaudhari17}, analyzing neural/protein networks, determining species \textbf{essential} for health of the environment \cite{allesina2009googling}, or even quantifying the \textbf{scientific impact} of researchers \cite{rank-senanayake15}.


%% RELATED WORK
PageRank is used in image search to identify canonical to display as a visual summary of a larger set of images returned from an image search engine \cite{rank-gleich15}.

Kolda and Procopio \cite{kolda2009generalized} propose generalized models (based on BadRank \cite{sobek2002pr0} and TrustRank \cite{rank-gyongyi04}) to combat spam sites that create no new information content, but attempt to capture Google search results by appearing to contain information, and include the idea of adding self-links to all nodes in the graph, to fix the dangling nodes. For spam-link applications, this method of handling dangling nodes is superior --- in a modeling sense --- to the alternatives.

Chepelianskii \cite{chepelianskii2010towards} uses the distribution of PageRank and reverse PageRank scores to characterize the properties of the linux kernel (a software system), using a call graph of the linux kernel.

Zuo et al. \cite{zuo2012network} use PageRank to evaluate the importance of brain regions given observed correlations of brain activity.

% PageRank \cite{rank-page99} is an algorithm that measures the importance of nodes in a network by assigning numerical scores based on the structure of links. It finds applications in web page ranking, identifying misinformation, predicting traffic flow, and protein target identification. The increasing availability of vast amounts of data represented as graphs has led to a significant interest in parallel algorithms for computing PageRank \cite{rank-garg16, rank-nvgraph, rank-giri20, rank-sarma13}.\ignore{--- it has been implemented on multicore CPUs \cite{rank-garg16}, GPUs \cite{rank-nvgraph}, FPGAs \cite{rank-guoqiang20}, SpMV ASICs \cite{rank-sadi18}, CPU-GPU hybrids \cite{rank-giri20}, CPU-FPGA hybrids \cite{rank-li21}, and distributed systems \cite{rank-sarma13}.}

However, most real-world graph evolve with time. Here, frequent edge insertions and deletions make recomputing PageRank from scratch impractical, particularly for small, rapid changes. Existing strategies optimize by iterating from the prior snapshot's ranks, reducing the number of iterations needed for convergence. For further improvements, it is essential to recompute only the ranks of vertices likely to change. A prevalent approach involves identifying reachable vertices from the updated regions of the graph, and limiting processing to such vertices. However, if updates are randomly distributed, they often fall within dense graph regions, necessitating processing of a substantial portion of the graph.

To reduce computational effort, one can incrementally expand the set of affected vertices starting from the updated graph region, rather than processing all reachable vertices from the first iteration. Additionally, it is possible to skip processing a vertex's neighbors if the change in its rank is small and is expected to have minimal impact on the ranks of its neighboring vertices. This technical report introduces such an approach.




\subsection{Our Contributions}

This report introduces our Dynamic Frontier approach\footnote{\url{https://github.com/puzzlef/pagerank-openmp-dynamic}}, which, when given a batch update involving edge insertions and deletions, incrementally identifies affected vertices likely to undergo rank changes with minimal overhead. On a server equipped with a 64-core AMD EPYC-7742 processor, our Dynamic Frontier PageRank surpasses Static, Naive-dynamic, and Dynamic Traversal PageRank by $7.8\times$, $2.9\times$, and $3.9\times$ respectively, for uniformly random batch updates of size $10^{-7}|E|$ to $10^{-3}|E|$, where $|E|$ is the number of edges in the original graph. Additionally, our approach exhibits a performance improvement of $1.8\times$ for each doubling of threads.




%% - Use --- for a dash.
%% - Use ``camera-ready'' for quotes.
%% - Use {\itshape very} or \textit{very} for italicized text.
%% - Use \verb|acmart| or {\verb|acmart|} for mono-spaced text.
%% - Use \url{https://capitalizemytitle.com/} for URLs.
%% - Use {\bfseries Do not modify this document.} for important boldface details.
%% - Use \ref{fig:name} for referencing.

%% For a block of pre-formatted text: 
% \begin{verbatim}
%   \renewcommand{\shortauthors}{McCartney, et al.}
% \end{verbatim}

%% For a list of items:
% \begin{itemize}
% \item the ``ACM Reference Format'' text on the first page.
% \item the ``rights management'' text on the first page.
% \item the conference information in the page header(s).
% \end{itemize}

%% For a table:
% \begin{table}
%   \caption{Frequency of Special Characters}
%   \label{tab:freq}
%   \begin{tabular}{ccl}
%     \toprule
%     Non-English or Math&Frequency&Comments\\
%     \midrule
%     \O & 1 in 1,000& For Swedish names\\
%     $\pi$ & 1 in 5& Common in math\\
%     \$ & 4 in 5 & Used in business\\
%     $\Psi^2_1$ & 1 in 40,000& Unexplained usage\\
%   \bottomrule
% \end{tabular}
% \end{table}

%% For a full-width table:
% \begin{table*}
%   \caption{Some Typical Commands}
%   \label{tab:commands}
%   \begin{tabular}{ccl}
%     \toprule
%     Command &A Number & Comments\\
%     \midrule
%     \texttt{{\char'134}author} & 100& Author \\
%     \texttt{{\char'134}table}& 300 & For tables\\
%     \texttt{{\char'134}table*}& 400& For wider tables\\
%     \bottomrule
%   \end{tabular}
% \end{table*}


%% For inline math:
% \begin{math}
%   \lim_{n\rightarrow \infty}x=0
% \end{math},

%% For a numbered equation:
% \begin{equation}
%   \lim_{n\rightarrow \infty}x=0
% \end{equation}

%% For an unnumbered equation:
% \begin{displaymath}
%   \sum_{i=0}^{\infty} x + 1
% \end{displaymath}

%% For a figure:
% \begin{figure}[h]
%   \centering
%   \includegraphics[width=\linewidth]{inc/sample-franklin}
%   \caption{1907 Franklin Model D roadster. Photograph by Harris \&
%     Ewing, Inc. [Public domain], via Wikimedia
%     Commons. (\url{https://goo.gl/VLCRBB}).}
%   \Description{A woman and a girl in white dresses sit in an open car.}
% \end{figure}

%% For a teaser figure.
% \begin{teaserfigure}
%   \includegraphics[width=\textwidth]{sampleteaser}
%   \caption{figure caption}
%   \Description{figure description}
% \end{teaserfigure}


\section{Related work}
\label{sec:related}
Early work in dynamic graph algorithms in the sequential setting includes the sparsification method proposed by Eppstein et al. \cite{graph-eppstein97} and Ramalingam's bounded incremental computation approach \cite{incr-ramalingam96}.\ignore{The latter advocates measuring the work done as part of the update in proportion to the effect the update has on the computation.} Several approaches have been suggested for incremental computation of approximate PageRank values in a dynamic or evolving graph. Chien et al. \cite{rank-chien01} identify a small region near updated vertices in the graph and represent the rest of the graph as a single vertex in a smaller graph. PageRanks are computed for this reduced graph and then transferred back to the original graph. Chen et al. \cite{chen2004local} propose various methods to estimate the PageRank score of a webpage using a small subgraph of the entire web, by expanding backwards from the target node along reverse hyperlinks. Bahmani et al. \cite{bahmani2010fast} analyze the efficiency of Monte Carlo methods for incremental PageRank computation. Zhan et al. \cite{zhan2019fast} introduce a Monte Carlo-based algorithm for PageRank tracking on dynamic networks, maintaining $R$ random walks starting from each node. Pashikanti et al. \cite{rank-pashikanti22} also employ a similar Monte Carlo-based approach for updating PageRank scores upon vertex and edge insertions/deletions.

A few approaches have been devised to update exact PageRank scores on dynamic graphs. Zhang \cite{rank-zhang17} introduces a simple incremental PageRank computation system for dynamic graphs, which we refer to as the \textit{Naive-dynamic} approach, on hybrid CPU and GPU platforms\ignore{ --- employing the Update-Gather-Apply-Scatter (UGAS) computation model}. Additionally, Ohsaka et al. \cite{ohsaka2015efficient} propose a method for locally updating PageRank using the Gauss-Southwell method, prioritizing the vertex with the greatest residual for initial updating; however, their algorithm is inherently sequential. A widely adopted approach for updating PageRank \cite{rank-desikan05, kim2015incremental, rank-giri20, sahu2022dynamic} is based on the observation that changes in the out-degree of a node do not influence its PageRank score, adhering to the first-order Markov property. The portion of the graph undergoing updates, involving edge insertions or deletions, is used to identify the affected region of the graph in a preprocessing step. This is typically accomplished through Breadth-First Search (BFS) or Depth-First Search (DFS) traversal from vertices connected to the inserted or deleted edges. Subsequently, PageRanks are computed solely for this region. Desikan et al. \cite{rank-desikan05} originally proposed this, which we term as the \textit{Dynamic Traversal} approach in this report. Kim and Choi \cite{kim2015incremental} apply this approach with an asynchronous PageRank implementation, while Giri et al. \cite{rank-giri20} utilize it with collaborative executions on multi-core CPUs and massively parallel GPUs. Sahu et al. \cite{sahu2022dynamic} employ this strategy on a Strongly Connected Component (SCC)-based graph decomposition to limit computation to reachable SCCs from updated vertices, on multi-core CPUs and GPUs.

In our previous study \cite{sahu2024incrementally}, we introduced an incrementally expanding method for updating PageRank on dynamic graphs, demonstrating strong performance on dynamic graphs derived from large static graphs with uniformly random batch updates. However, we noted that the approach did not perform as effectively on real-world dynamic graphs. Adjusting parameters, specifically lowering the frontier tolerance, was necessary to achieve decent performance. Therefore, the selection of frontier tolerance, along with the method of frontier expansion, relies on the nature of batch updates.\ignore{In this technical report, we explore how to expand the frontier, along with the choice of a suitable frontier tolerance, for real world dynamic graphs. Next, we explore how to prune processed vertices, i.e., contract the set of affected vertices, and select an appropriate prune tolerance. Our experiments on both real-world dynamic graphs and large static graphs with uniformly random batch updates demonstrate the effectiveness of our approach with minor adjustments in both scenarios. Note that a batch update refers to a set of simultaneous changes, i.e., edge insertions or deletions, applied to the graph in a single step.}

Further, Bahmani et al. \cite{rank-bahmani12} introduce an algorithm for selectively crawling a small section of the web to estimate the true PageRank of the graph at a given moment, while Berberich et al. \cite{rank-berberich07} propose a method to compute normalized PageRank scores that remain robust against non-local changes in the graph. These approaches diverge from our improved \textit{Dynamic Frontier} approach, which concentrates on computing the PageRank vector itself rather than on the tasks of web crawling or maintaining normalized scores.
%% Add other interesting variations?




%% COMPRE
%% ------
% Note that as originally conceived, the PageRank model does not factor a web browser's \textbf{back button} into a surfer's hyperlinking possibilities. Surfers in one class, if teleporting, may be much more likely to jump to pages about sports, while surfers in another class may be much more likely to jump to pages pertaining to news and current events. Such differing teleportation tendencies can be captured in two different \textbf{personalization vectors}. However, it makes the once query-independent, user independent PageRankings, user-dependent and more calculation-laden. Nevertheless, it seems this little personalization vector has had more significant side effects. This personalization vector, along with a \textbf{non-uniform/weighted} version of PageRank \cite{pr-dubey16} can help control spamming done by the so-called link farms \cite{pr-deeper01}.

% Techniques to optimize the PageRank algorithm usually fall in two categories. One is to try \textbf{reducing the work per iteration}, and the other is to try \textbf{reducing the number of iterations}. Often, these goals are at odds against each other. The \textbf{adapting PageRank technique} "locks" vertices which have converged, and saves iteration time by skipping their computation \cite{pr-deeper01}. Identical nodes, which have the same in-links, can be removed to reduce duplicate computations and thus reduce iteration time. Road networks often have chains which can be short-circuited before PageRank computation to improve performance. Final ranks of chain nodes can be easily calculated. This reduces both the iteration time, and the number of iterations. If a graph has no dangling nodes, PageRank of each strongly connected component can be computed in topological order. This helps reduce the iteration time, number of iterations, and also enable concurrency in PageRank computation. The combination of all of the above methods is the \textbf{STIC-D algorithm} (see Figure \ref{fig:about-pagerank-sticd}) \cite{pr-sticd16}. A somewhat similar aggregation algorithm is \textbf{BlockRank} which computes the PageRank of hosts, local PageRank of pages within hosts independently, and aggregates them with weights for the final rank vector. The ranks of vertices for the entire graph can be found efficiently by computing the sub-PageRank of each connected component, and then using the sub-PageRanks together to form the global PageRank (Avrachenkov et. al. \cite{pr-avrachenkov04}). These methods exploit the inherent reducibility in the graph. The \textbf{Jacobi method} can also be used to compute the PageRank vector (Bianchini et. al. \cite{pr-bianchini05}) \cite{pr-deeper01}. \textbf{Monte Carlo} based PageRank methods consider several random walks on the input graph to get approximate PageRanks. Its optimizations for distributed PageRank computation (specially for undirected graphs) \cite{compute-frey13}, map-reduce algorithm for personalized PageRank \cite{pr-bahmani11}, and reordering strategy (to reduce space and compute complexity on GPU) for local PageRank \cite{pr-lai17} are present.

% The time per iteration can be reduced further by taking note of the fact that the traditional algorithm is \textbf{not computationally bound} and \textbf{generates fine granularity random accesses} (it exhibits irregular parallelism). This causes poor memory bandwidth and compute utilization, and the extent of this is quite dependent upon the graph structure \cite{compute-hunold15} \cite{pr-lakhotia18}. \textit{Four strategies for neighbour iteration} were attempted, to help reason about the \textit{expected impact of a graph's structure} on the performance of each strategy \cite{compute-hunold15}. CPUs/GPUs are generally designed optimized to \textbf{load memory} in blocks (cache-lines in CPUs, coalesced memory reads in GPUs), and not for fine-grained accesses. Being able to adjust this behaviour depending upon application (PageRank) can lead to performance improvements. Techniques like \textit{prefetching to SRAM, using a high-performance shuffle network} \cite{graph-wang15}, \textit{indirect memory prefetcher (of the form $A[B[i]]$), partial cache line accessing mechanisms} \cite{memory-yu15}, \textit{adjusting data layout} \cite{pr-lakhotia18} \textit{(for sequential DRAM access} \cite{pr-zhou15} \textit{), and branch avoidance mechanisms (with partitioning)} \cite{pr-lakhotia18} are used. Large graphs can be \textbf{partitioned }or decomposed into subgraphs to help reduce cross-partition data access that helps both in distributed, as well as shared memory systems (by reducing random accesses). Techniques like \textit{chunk partitioning} \cite{pr-rungsawang12}, \textit{cache/propagation blocking} \cite{pr-beamer17}, \textit{partition-centric processing with gather-apply-scatter model} \cite{pr-lakhotia18}, \textit{edge-centric scatter-gather with non-overlapping vertex-set} \cite{pr-zhou17}, \textit{exploiting node-score sparsity} \cite{pr-li21}, and even \textit{personalized PageRank based partitioning} \cite{pr-mazaheri19} have been used. Graph/subgraph \textbf{compression} can also help reduce memory bottlenecks \cite{pr-rungsawang12} \cite{pr-guoqiang20}, and enable processing of larger graphs in memory. A number of techniques can be used to compress adjacency lists, such as, \textit{delta encoding of edge/neighbour ids} \cite{graph-bharat98}, and \textit{referring sets of edges in edge lists} \cite{graph-adler01} \cite{graph-raghavan03} (hard to find reference vertices though) \cite{pr-deeper01}. Since the rank vector (possibly even including certain additional page-importance estimates) must reside entirely in main memory, a few compression techniques have been attempted. These include \textit{lossy encoding schemes based on scalar quantization} seeking to minimize the distortion of search-result rankings \cite{pr-haveliwala02} \cite{pr-deeper01}, and using \textit{custom half-precision floating-point formats} \cite{pr-molahosseini20}.

% As new software/hardware \textbf{platforms} appear on the horizon, researchers have been eager to test the performance of PageRank on the hardware. This is because each platform offers its own unique architecture and engineering choices, and also because PageRank often serves as a good benchmark for the capability of the platform to handle various other graph algorithms. Attempts have been made on distributed frameworks like \textit{Hadoop} \cite{pr-abdullah10}, and even \textit{RDBMS} \cite{compute-barolli21}. A number of implementations have been done on \textit{standard multicores} \cite{compute-barolli21}, \textit{Cell BE} \cite{compute-buehrer08} \cite{pr-zhou17}, \textit{AMD GPUs} \cite{pr-wu10}, \textit{NVIDIA/CUDA GPUs} \cite{pr-bisson16} \cite{pr-zhou17} \cite{graph-seo15}, \textit{GPU clusters} \cite{pr-rungsawang12}, \textit{FPGAs} \cite{pr-mughrabi21} \cite{graph-wang15} \cite{pr-guoqiang20}, \textit{CPU-FPGA hybrids} \cite{pr-hassan21} \cite{pr-usta21} \cite{pr-li21}, and even on SpMV \textit{ASICs} \cite{pr-sadi18}.

% PageRank algorithm is a \textbf{live algorithm} which means that an ongoing computation can be paused during graph update, and simply be resumed afterwards (instead of restarting it). The first \textbf{updating} paper by Chien et al. \cite{pr-chien01} identifies a tiny region of the graph near the changed vertices/edges and model the remainder of the graph as a single vertex in a new, much smaller graph. PageRanks are computed for the small graph and then transferred to the original graph \cite{pr-deeper01}. A common technique used for dynamic PageRank algorithm, given a small change to the input graph, is to find the affected region in the preprocessing step with breadth-first search (BFS) or depth-first search (DFS) traversal from the vertices connecting the edges that were inserted or deleted \cite{pr-desikan05} \cite{pr-giri20}.

  % 1 &
  % Prasanna Desikan, Nishith Pathak, Jaideep Srivastava, and Vipin Kumar. 2005. \textbf{Incremental page rank computation on evolving graphs}. In Special interest tracks and posters of the 14th international conference on World Wide Web (WWW '05). Association for Computing Machinery, New York, NY, USA, 1094–1095.\linebreak
  % DOI: https://doi.org/10.1145/1062745.1062885 &
  % 54 \\ \hline
  % \multicolumn{3}{|p{14cm}|}{This paper describes a simple method for computing dynamic pagerank, based on the fact that change of out-degree of a node does not affect its pagerank (first order markov property). The part of the graph which is updated (edge additions / edge deletions / weight changes) is used to find the affected partition of the graph using BFS. The unaffected partition is simply scaled, and pagerank computation is done only for the affected partition. \footnote{https://gist.github.com/wolfram77/f0a7534d49d5c07d4479ec3966c5d635}} \\ \hline

  % 2 &
  % Yen-Yu Chen, Qingqing Gan, and Torsten Suel. 2002. \textbf{I/O-efficient techniques for computing pagerank}. In Proceedings of the eleventh international conference on Information and knowledge management (CIKM '02). Association for Computing Machinery, New York, NY, USA, 549–557.\linebreak
  % DOI: https://doi.org/10.1145/584792.584882 &
  % 33 \\ \hline
  % \multicolumn{3}{|p{14cm}|}{This paper describes a technique to partition the link file of the whole file into blocks of a range of destination nodes, with partial source nodes, so that it is possible to run power iteration of pagerank of massive graphs which do not fit in memory. The graphs must be stored on disk, and partitions of the graphs are scanned in every iteration until the ranks converge. Unlike Haveliwala's technique, this is similar to pull based pagerank. Both methods have similarities with join techniques in database systems. Topic-sensitive pagerank is also discussed which finds pagerank of graphs related to a specific keywords beforehand, and merges them together based upon the query (might return better results than global pagerank). This requires small adjustments to the random jump probability factor $(1-d)$. \footnote{https://gist.github.com/wolfram77/925cede0214aa0f391f34fa8ce137290}} \\ \hline

  % 3 &
  % Paritosh Garg and Kishore Kothapalli. 2016. \textbf{STIC-D: algorithmic techniques for efficient parallel pagerank computation on real-world graphs}. In Proceedings of the 17th International Conference on Distributed Computing and Networking (ICDCN '16). Association for Computing Machinery, New York, NY, USA, Article 15, 1–10.\linebreak
  % DOI: https://doi.org/10.1145/2833312.2833322 &
  % 7 \\ \hline
  % \multicolumn{3}{|p{14cm}|}{In this paper, the authors exploit the reducibility of dead-end free graphs to compute PageRanks. They first split the vertices into strongly connected components (SCCs) and represent each SCC as a vertex in a block-graph. Each SCC is then processed as per its topological order in the block-graph. This enables them to reduce the operating memory requirement, thanks to the smaller size of SCCs that are processed in one go. As SCCs are processed in topological order, unnecessary iterations on vertices that are unlikely to converge are avoided. Processing vertices grouped into SCCs also improves performance due to inherent spatial locality within an SCC. In addition, this method allows SCCs residing on the same level in the block-graph to be processed independently of each other. This is demonstrated by the authors by processing each such SCC in parallel with OpenMP. They also present three algorithmic techniques for eliminating redundancies in PageRank computation, namely skipping repeated computation on in-identical vertices (minimize redundant computation), short circuiting chain vertices (help accelerate convergence), and skipping computation on vertices that appear to have converged (minimize redundant computation).The suitability of these techniques depend upon the nature of input graph. They study the techniques on four classes of real-world graphs: web graphs, social networks, citation and collaboration networks, and road networks. Their implementation achieves an average speedup of 32\% compared to a baseline implementation. \footnote{https://gist.github.com/wolfram77/bb09968cc0e592583c4b180243697d5a}} \\ \hline

  % 4 &
  % Stergios Stergiou. 2020. \textbf{Scaling PageRank to 100 Billion Pages}. Proceedings of The Web Conference 2020. Association for Computing Machinery, New York, NY, USA, 2761–2767.\linebreak
  % DOI: https://doi.org/10.1145/3366423.3380035 &
  % 1 \\ \hline
  % \multicolumn{3}{|p{14cm}|}{In this paper, the author exploits the fact the communication required between iterations is identical. He uses this to develop a new communication format that allows significant reduction in bandwidth requirement. He experiments on massive web graphs with up to 38 billion vertices and 3.1 trillion edges, requiring a per-iteration time of 34.4 seconds, which is more than an order of magnitude improvement over the state-of-the-art. \footnote{https://gist.github.com/wolfram77/10964cd26f11f7a7299e7b74a0be7e7e}} \\ \hline
%% ------

% Coarse PageRank approaches: Arasu et al. \cite{rank-arasu02} proposed HostRank, where the web is aggregated at the level of host names. BlockRank \cite{kamvar2003exploiting} uses HostRank to initialize PageRank, while DirRank \cite{rank-eiron04} forms aggregation at the level of directories of websites.

%% We use an asynchronous approach:
% Real-Time PageRank on Dynamic Graphs (2023): In this paper, Sallinen et al. \cite{sallinen2023real} compute PageRank asynchronously for real-time, on demand PageRank computation with arbitrary granularity. They model PageRank as a flow problem, where mass is absorbed by the page, and the rest is distributed to neighbors. This is done by sending delta values of probability mass depending on edge deletion or insertions by adjustment upon earlier values. Sink/dangling vertices (dead ends) are handled as usual (teleport).

%% Interesting approach:
% PageRank Algorithm Based on Dynamic Damping Factor (2023): Existing methods often set the damping factor empirically, overlooking the relevance of web visitors’ topics. HaoLin et al. \cite{haolin2023pagerank} propose an adaptive dynamic damping factor based on the web browsing context, and demonstrate that it effectively mitigates the impact of noisy web pages on query results and improves the convergence speed.

%% Sliding window approach.
% Time-Aware Ranking in Dynamic Citation Networks (2011): In this paper, Ghosh et al. \cite{ghosh2011time} consider the temporal order of links and chains of links connecting to a node with some temporal decay factors, and show that it is more appropriate for predicting future citations and PageRank scores with regard to new citations.

%% Other interesting approach:
% A Dynamical System for PageRank with Time-Dependent Teleportation (2014): In this paper, Gleich and Rossi \cite{gleich2014dynamical} propose a time-dependent teleportation to the PageRank score. The magnitude of the deviation from a static PageRank vector is given by a PageRank problem with complex-valued teleportation parameters. They demonstrate the utility of dynamic teleportation on both the article graph of Wikipedia, where the external interest information is given by the number of hourly visitors to each page, and the Twitter social network, where external interest is the number of tweets per month. They show that using information from the dynamical system helps improve a prediction task and identify trends in the data.

%% Other interesting approach:
% Temporal PageRank (2016): In this paper, Rozenshtein and Gionis \cite{rozenshtein2016temporal} propose an extension of static PageRank to temporal PageRank so that only temporal walks are considered instead of all possible walks. In order to compute temporal PageRank we need to process the sequence of interactions and calculate the weighted number of temporal walks. Their algorithm counts explicitly the weighted number of temporal walks.

%% Similar to STIC-D:
% Divide and conquer approach for efficient pagerank computation (2006): In this paper, Desikan et al. \cite{desikan2006divide} propose a graph-partitioning technique for PageRank, on which computation can be performed independently.

%% Similar to STIC-D:
% A componentwise PageRank algorithm (2015): In this paper, Engstrom et al. \cite{engstrom2015componentwise} propose two PageRank algorithms, one similar to the Lumping algorithm proposed by Qing et al. which handles certain types of vertices faster, and last, another PageRank algorithm which can handle more types of vertices as well as strongly connected components more effectively. This is similar to the work of Garg et al.

%% Streaming PageRank:
% Estimating PageRank on graph streams (2011): In this paper, Sarma et al. \cite{rank-sarma11} study the streaming model for PageRank, which uses a small amount of memory (preferably sub-linear in the number of nodes n). They compute approximate PageRank values in Õ(nM−1/4) space and Õ(M3/4) passes. They also give another approach to approximate the PageRank values in just Õ(1) passes although this requires Õ(nM) space.

%% Applications of PageRank:
% PageRank Tracker: From Ranking to Tracking (2013): PageRank has been used by Gong et al. \cite{gong2013pagerank} in video object tracking to improve its robustness, i.e., to address difficulties with adaptation to environmental or target change. Determining the target is equivalent to finding the unlabeled sample that is the most associated with the existing labeled set.

%% Applications of PageRank:
% Abstracting PageRank To Dynamic Asset Valuation (2006): In this paper, Sawilla \cite{sawilla2006abstracting} uses (weighted) PageRank to quickly and dynamically calculate a relative value for all assets in an organization in any context in which dependencies may be specified. Their scheme works in general and will provide asset valuation in any context, be it confidentiality, integrity, availability, or even political capital.

%% For introduction, also a bit here:
% Adaptive Implementation to Update Page Rank on Dynamic Networks (2021): In this oral presentation, Srinivasan \cite{srinivasan2021adaptive} talk about the fact that There are a lot of attempts made to parallelize the page rank algorithm for static networks, however, there are only very few attempts made to compute page rank on dynamic networks. As the networks change with time, computing page rank or updating is an expensive operation, the previous attempts have only approximated the metric to avoid recomputation. In this paper, we introduce a framework where we try to update the page rank of the vertices which embraces change as the network changes. The proposed framework is implemented on a shared memory system and experiments on real-world and synthetic networks show good scalability. The framework proposed gets an input set of networks, initial page rank values for all the vertices, and a set of batches that has the changeset. As the batches are processed in parallel, affected vertices are identified and marked for an update, once the batch is processed the vertices affected or identified their page rank values are computed. The main contribution of this paper is the proposed framework avoids recomputation of all vertices, and only recomputes few vertices, and avoids approximation to provide accurate values.


\section{Preliminaries}
\label{sec:preliminaries}
\subsection{PageRank algorithm}
\label{sec:pagerank}

The PageRank, denoted as $R[v]$, of a vertex $v \in V$ in the graph $G(V, E)$, quantifies its \textit{importance} based on the number and significance of incoming links. Equation \ref{eq:pr} outlines the computation of PageRank for vertex $v$ in graph $G$, where $V$ represents the set of vertices\ignore{($N = |V|$)}, $E$ represents the set of edges\ignore{($M = |E|$)}, $G.in(v)$ denotes the incoming neighbors of vertex $v$, $G.out(v)$ denotes the outgoing neighbors of vertex $v$, and $\alpha$ represents the damping factor. Initially, each vertex has a PageRank of $1/|V|$. The \textit{power-iteration} method iteratively updates these values until they converge within a specified tolerance $\tau$. This is typically measured using the $L_1$-norm \cite{ohsaka2015efficient}, though $L_2$ and $L_\infty$-norm are also occasionally used.

The random surfer model, integral to the PageRank algorithm, conceptualizes a surfer navigating the web by following links on each page. The damping factor $\alpha$, with a default value of $0.85$, represents the probability that the surfer continues along a link instead of jumping randomly. PageRank for each page reflects the long-term likelihood of the surfer visiting that page, based on starting from a random page and following links\ignore{according to the damping factor}. PageRank values are essentially the eigenvector of a transition matrix, which encodes probabilities of moving between pages in a Markov Chain.

Dead ends, also known as dangling vertices, pose a challenge in PageRank computation. They are vertices with no out-links, ans thus force the surfer to jump to a random web page. Consequently, dead ends contribute their rank equally among all vertices in the graph --- this must be computed in each iteration, and is therefore an overhead. We address this issue by adding self-loops to all vertices in the graph \cite{kolda2009generalized, rank-andersen07, rank-langville06}. In a streaming environment, this option may be the most suitable. It has also been observed to be superior in spam-link applications \cite{kolda2009generalized}.

% In order to understand the PageRank algorithm, consider the \textbf{random surfer model} on a graph with several vertices and interconnecting edges. The surfer (such as you) initially visits a vertex at random. He then follows one of the edges leading to another vertex. After following some edges, the surfer would eventually decide to visit another vertex (at random). The probability of the random surfer being on a certain vertex is what the PageRank algorithm returns. This probability (or importance) of a vertex depends upon the importance of vertices pointing to it. This definition of PageRank is recursive, and takes the form of an \textbf{eigen-value problem}. Solving for PageRank thus requires multiple iterations of computation, which is known as the \textbf{power-iteration method}. Each computation is essentially a \textbf{(sparse) matrix multiplication}. A damping factor (of 0.85) is used to counter the effect of \textbf{spider-traps} (like self-loops), which can otherwise suck up all importance. \textbf{Dead-ends} (vertices with no out-links) are countered by effectively linking it to all vertices of the graph, which otherwise would leak out importance \cite{pr-leskovec19}. See Figure \ref{fig:about-pagerank} for example. The procedure to obtain such ranks is shown in Algorithm \ref{alg:pr-static}.

% (\textbf{Markov chain}) (making Markov matrix column stochastic)

% In order to understand the PageRank algorithm, consider this \textbf{random (web) surfer model}. Each web page is modelled as a vertex, and each hyperlink as an edge. The surfer (such as you) initially visits a web page at random. He then follows one of the links on the page, leading to another web page. After following some links, the surfer would eventually decide to visit another web page (at random). The probability of the random surfer being on a certain page is what the PageRank algorithm returns. This probability (or importance) of a web page depends upon the importance of web pages pointing to it (\textbf{Markov chain}). This definition of PageRank is recursive, and takes the form of an \textbf{eigen-value problem}. Solving for PageRank thus requires multiple iterations of computation, which is known as the \textbf{power-iteration method}. Each computation is essentially a \textbf{(sparse) matrix multiplication}. A damping factor (of 0.85) is used to counter the effect of \textbf{spider-traps} (like self-loops), which can otherwise suck up all importance. \textbf{Dead-ends} (web pages with no out-links) are countered by effectively linking it to all vertices of the graph (making Markov matrix column stochastic), which otherwise would leak out importance \cite{pr-leskovec19}. See \ref{fig:about-pagerank} for example. The procedure to obtain such ranks is shown in algorithm \ref{alg:pr-static}.

\begin{equation}
\label{eq:pr}
    R[v] = \alpha \times \sum_{u \in G.in(v)} \frac{R[u]}{|G.out(u)|} + \frac{1 - \alpha}{n}
\end{equation}




\subsection{Dynamic Graphs}
\label{sec:about-dynamic}

A dynamic graph can be conceptualized as a sequence of graphs, where $G^t(V^t, E^t)$ represents the graph at time step $t$. The changes between consecutive time steps $t-1$ and $t$, from $G^{t-1}(V^{t-1}, E^{t-1})$ to $G^t(V^t, E^t)$, can be represented as a batch update $\Delta^t$ at time step $t$. This update comprises a set of edge deletions $\Delta^{t-}$, defined as $\{(u, v)\ |\ u, v \in V\} = E^{t-1} \setminus E^t$, and a set of edge insertions $\Delta^{t+}$, defined as $\{(u, v)\ |\ u, v \in V\} = E^t \setminus E^{t-1}$.


\paragraph{Interleaving graph updates with computation:}

We assume changes to the graph to be batched, with updating of the graph and algorithm execution occurring in an interleaved manner --- allowing only one writer on the graph structure at any given time. If it is needed to update the graph in parallel with the computation, a graph snapshot needs to be obtained, on which the computation can be performed. See for example, the Aspen graph processing framework, which minimizes\ignore{read-only} snapshot acquisition costs \cite{graph-dhulipala19}.




\subsection{Existing approaches for updating PageRank on Dynamic Graphs}

\subsubsection{Naive-dynamic approach}
\label{sec:about-naive}

This\ignore{straightforward} approach involves updating vertex ranks in dynamic networks by initializing them with ranks from the previous graph snapshot and running the PageRank algorithm on all vertices. Rankings obtained using this approach are at least as accurate as those obtained from the static algorithm.\ignore{Zhang et al. \cite{rank-zhang17} have explored the \textit{Naive-dynamic} approach in the hybrid CPU-GPU setting.}
% We also include the pseudocode for \textit{Naive-dynamic With-barrier} PageRank (\NaiWbar{}) for reference in Algorithm \ref{alg:with-barrier-naive-dynamic}.


\subsubsection{Dynamic Traversal approach}
\label{sec:about-traversal}

Initially proposed by Desikan et al. \cite{rank-desikan05}, this approach involves skipping the processing of vertices whose ranks are unlikely to be updated by the given batch update. For each edge deletion or insertion $(u, v)$ in the batch update, all vertices reachable from vertex $u$ in either graph $G^{t-1}$ or $G^t$ are marked as affected, using DFS or BFS.\ignore{Giri et al. \cite{rank-giri20} have explored the \textit{Dynamic Traversal} approach in the hybrid CPU-GPU setting. On the other hand, Banerjee et al. \cite{rank-sahu22} have explored this approach in the CPU and GPU settings separately where they compute the ranks of vertices in topological order of strongly connected components (SCCs) to minimize unnecessary computation. They borrow this ordered processing of SCCs from the original static algorithm proposed by Garg et al. \cite{rank-garg16}.}
% Additionally, we provide the pseudocode for \textit{Dynamic Traversal With-barrier} PageRank (\TraWbar{}), which can be referred to in Algorithm \ref{alg:with-barrier-dynamic-traversal}.


\section{Approach}
\label{sec:approach}
\subsection{Our improved Dynamic Frontier (DF) approach for Real-world Dynamic Graphs}
\label{sec:frontier}

If a batch update $\Delta^{t-} \cup \Delta^{t+}$ is small compared to the total number of edges $|E|$, then it is expected that the ranks of only a few vertices change. Our proposed Dynamic Frontier approach incorporates this aspect, and identifies affected vertices via an incremental process. This allows it to avoid unnecessary computation, since ranks of vertices far for the updated region of the graph cannot have a change in their ranks until the ranks of its immediate in-neighbors change. In addition, we avoid marking the neighbors of a vertex as affected, if the change in rank of the vertex is small enough and is likely to have minimal effect on the ranks of its neighbors.


\subsubsection{Explanation of the approach}
\label{sec:frontier-explanation}

Consider a batch update consisting of edge deletions $(u, v) \in \Delta^{t-}$ and insertions $(u, v) \in \Delta^{t+}$. We first initialize the rank of each vertex to that obtained in the previous snapshot of the graph.

\begin{figure*}[hbtp]
  \centering
  \subfigure[Initial graph]{
    \label{fig:about-frontier-01}
    \includegraphics[width=0.23\linewidth]{out/about-frontier-01.pdf}
  }
  \subfigure[Marking affected (initial)]{
    \label{fig:about-frontier-02}
    \includegraphics[width=0.23\linewidth]{out/about-frontier-02.pdf}
  }
  \subfigure[After first iteration]{
    \label{fig:about-frontier-03}
    \includegraphics[width=0.23\linewidth]{out/about-frontier-03.pdf}
  }
  \subfigure[After second iteration]{
    \label{fig:about-frontier-04}
    \includegraphics[width=0.23\linewidth]{out/about-frontier-04.pdf}
  } \\[-2ex]
  \caption{An example showcasing our improved \textit{Dynamic Frontier} approach. The initial graph has $12$ vertices and $16$ edges. The graph is updated with an edge insertion $(6, 8)$ and an edge deletion $(2, 1)$. Consequently, the outgoing neighbors of vertices $2$ and $6$ (i.e., vertices $1$, $3$, $8$, and $12$) are marked as affected (highlighted in yellow). In the first iteration, when computing the ranks of these affected vertices, it is observed that the relative change in rank of vertices $1$, $3$, $8$, and $12$ exceeds the frontier tolerance $\tau_f$ (indicated with a red border). Therefore, their outgoing neighbors (i.e., vertices $4$, $5$, $7$, $9$, $11$, and $12$) are also marked as affected. Vertices $1$, $3$, and $8$ are subsequently no longer marked as affected due to their relative rank change falling below the prune tolerance $\tau_p$. In the second iteration, the relative rank change of vertices $5$, $9$, $11$, and $12$ surpasses the frontier tolerance $\tau_f$, resulting in their outgoing neighbors (i.e., vertices $6$, $10$, and $11$) being marked as affected. Additionally, vertices $4$, $5$, $7$, $9$, and $12$ are no longer marked as affected as their relative rank change falls below prune tolerance $\tau_p$. In the following iteration, the rankings of affected vertices are updated once more. If the rank change of each vertex falls within the iteration tolerance $\tau$, indicating convergence, the algorithm terminates.}
  \label{fig:about-frontier}
\end{figure*}

\ignore{An example illustrating our improved \textit{Dynamic Frontier (DF)} approach. The initial graph has $12$ vertices and $16$ edges. The graph is updated with an edge insertion $(6, 8)$ and an edge deletion $(2, 1)$. Consequently, the outgoing neighbors of vertices $2$ and $6$ (i.e., vertices $1$, $3$, $8$, and $12$) are marked as affected (highlighted in yellow). In the first iteration, when computing the ranks of these affected vertices, it is observed that the change in rank of vertices $1$, $3$, $8$, and $12$ exceeds the frontier tolerance $\tau_f$ (indicated with a red border). Therefore, their outgoing neighbors (i.e., vertices $4$, $5$, $7$, $9$, $11$, and $12$) are also marked as affected. Vertices $1$, $3$, and $8$ are no longer marked as affected as the change in their rank is below the prune tolerance $\tau_p$. In the second iteration, the change in rank of vertices $5$, $9$, $11$, and $12$ exceeds the frontier tolerance $\tau_f$. Therefore, their outgoing neighbors (i.e., vertices $6$, $10$, and $11$) are marked as affected. Again, vertices $4$, $5$, $7$, $9$, $12$ are no longer marked as affected, as their change in rank falls below prune tolerance $\tau_p$. In the subsequent iteration, the ranks of affected vertices are again updated. If the change in rank of every vertex is within iteration tolerance $\tau$, the ranks of vertices have converged, and the algorithm terminates.}


\paragraph{Initial marking of affected vertex on edge deletion/insertion:}

For each edge deletion/insertion $(u, v)$, we initially mark the outgoing neighbors of the vertex $u$ in the previous $G^{t-1}$ and current graph snapshot $G^t$ as affected.

\paragraph{Incremental marking of affected vertices upon change in rank of a given vertex:}

Next, while performing PageRank computation, if the rank of any affected vertex $v$ changes in an iteration by an amount greater than the \textit{frontier tolerance} $\tau_f$, we mark its outgoing neighbors as affected. This process of marking vertices continues in every iteration.


\subsubsection{A simple example}

Figure \ref{fig:about-frontier} shows an example of the Dynamic Frontier approach. The initial graph, shown in Figure \ref{fig:about-frontier-01}, comprises $16$ vertices and $25$ edges. Subsequently, Figure \ref{fig:about-frontier-02} shows a batch update applied to the original graph involving the deletion of an edge from vertex $2$ to $1$ and the insertion of an edge from vertex $4$ to $12$. Following the batch update, we perform the initial step of the Dynamic Frontier approach, marking outgoing neighbors of $2$ and $4$ as affected, i.e., $1$, $3$, $4$, $8$, and $12$ are marked as affected (indicated with a yellow fill). Note that vertex $2$ is not affected as it is a source of the change while vertex $4$ being a neighbour of $2$ is marked as affected. Now, we are ready to execute the first iteration of PageRank algorithm.

During the first iteration (see Figure \ref{fig:about-frontier-03}), the ranks of affected vertices are updated. It is observed that the rank changes of vertices $1$ and $12$ surpass the frontier tolerance $\tau_f$ (highlighted with a red border). In response to this, we incrementally mark the outgoing neighbors of $1$ and $12$ as affected, i.e., vertices $3$, $5$, $11$, and $14$. 

During the second iteration (see Figure \ref{fig:about-frontier-04}), the ranks of affected vertices are again updated. Here, its is observed that the change in rank of vertices $3$, $5$, $11$, and $14$ is greater than frontier tolerance $\tau_f$. Thus, we mark the outgoing neighbors of $3$, $5$, $11$, and $14$ as affected, namely vertices $4$, $6$, and $15$. In the subsequent iteration, the ranks of affected vertices are again updated. If the change in rank of each vertex is within iteration tolerance $\tau$, the ranks of vertices have converged, and the algorithm terminates.




\subsection{Synchronous vs Asynchronous implementation}

In a synchronous implementation, separate input and output rank vectors are used, ensuring deterministic results for parallel algorithms through vector swapping at the end of each iteration. In contrast, an asynchronous implementation utilizes a single rank vector, potentially achieving faster convergence and eliminating memory copies for unaffected vertices in dynamic approaches\ignore{, but introduces non-deterministic results in parallel algorithms}.

To assess synchronous and asynchronous implementations for Dynamic Frontier PageRank, both are tested on batch updates (purely edge insertions) ranging from $10^{-7}|E|$ to $0.1|E|$ for Static, Naive-dynamic, Dynamic Traversal, and Dynamic Frontier PageRank. Figure \ref{fig:approach-async} depicts the average relative runtime of asynchronous implementations compared to their synchronous counterparts. Based on the results, we use the asynchronous implementations of Naive-dynamic, Dynamic Traversal, and Dynamic Frontier PageRank --- as they are faster, especially for smaller batch sizes.\ignore{This is due to a somewhat faster convergence and the absence of copy overhead (for Dynamic Traversal and Dynamic Frontier approaches).}




\subsection{Determination of Frontier tolerance ($\tau_f$)}

We now measure a suitable value for frontier tolerance $\tau_f$ that allows us to minimize the number of vertices we process (after marking them as affected), while ensuring that we obtain ranks with the desired tolerance, i.e. we obtain ranks with no higher error than Static PageRank for the same tolerance setting. For this, we adjust frontier tolerance $\tau_f$ from $\tau$ to $\tau / 10^5$ and obtain ranks of vertices with the Dynamic Frontier approach on batch updates (consisting purely of edge insertions) of size $10^{-7}|E|$ to $0.1|E|$.

Figure \ref{fig:adjust-frontier} illustrates the average relative runtime and rank error (in comparison to ranks obtained with reference Static PageRank) using the Dynamic Frontier approach. The figure suggests that as $\tau_f$ increases, runtime decreases, but it is accompanied by an increase in error. A frontier tolerance $\tau_f$ set at $\tau/10^4$ or $\tau/10^5$ yields ranks with lower error than Static PageRank, making them acceptable for uniformly random batch updates. To err on the side of caution, we opt for a frontier tolerance of $\tau_f = \tau/10^5$.

\begin{figure*}[!hbt]
  \centering
  \subfigure[Speedup with varying Frontier tolerance $\tau_f$]{
    \label{fig:adjust-frontier--speedup}
    \includegraphics[width=0.48\linewidth]{out/adjust-frontier-speedup.pdf}
  }
  \subfigure[Error in ranks obtained with varying Frontier tolerance $\tau_f$]{
    \label{fig:adjust-frontier--error}
    \includegraphics[width=0.48\linewidth]{out/adjust-frontier-error.pdf}
  } \\[-2ex]
  \caption{Average Speedup and Error in ranks obtained (with respect to ranks obtained with Reference Static PageRank) using \textit{Dynamic Frontier} approach, with frontier tolerance $\tau_f$ varying from $\tau$ to $\tau / 10^5$, on batch updates of size $10^{-7}|E|$ to $0.1|E|$. The figures indicate that increasing $\tau_f$ reduces runtime, but also increases the error. A Frontier tolerance $\tau_f$ of $\tau/10^4$ and $\tau/10^5$ obtain ranks with error lower than \textit{Static} PageRank, and are thus acceptable (we choose $\tau_f = \tau/10^5$ to be on the safe side).}
  \label{fig:adjust-frontier}
\end{figure*}

\begin{figure*}[!hbt]
  \centering
  \subfigure{
    \label{fig:adjust-prune--speedup}
    \includegraphics[width=0.48\linewidth]{out/adjust-prune-speedup.pdf}
  }
  \subfigure{
    \label{fig:adjust-prune--error}
    \includegraphics[width=0.48\linewidth]{out/adjust-prune-error.pdf}
  } \\[-2ex]
  \caption{Average Relative runtime with asynchronous implementations of \textit{Static}, \textit{Naive-dynamic}, \textit{Dynamic Traversal}, and \textit{Dynamic Frontier} approach compared to their respective synchronous implementations, on batch updates of size $10^{-7}|E|$ to $0.1|E|$ (right), and overall (left). The results indicate that asynchronous implementations are faster than synchronous ones, especially for smaller batch sizes. This is due to a somewhat faster convergence and the absence of copy overhead (for \textit{Dynamic Traversal} and \textit{Dynamic Frontier} approaches).}
  \label{fig:adjust-prune}
\end{figure*}

\begin{algorithm}[!hbt]
\caption{Our parallel Dynamic Frontier PageRank.}
\label{alg:frontier}
\begin{algorithmic}[1]
\Require{$G^{t-1}, G^t$: Previous, current input graph}
\Require{$\Delta^{t-}, \Delta^{t+}$: Edge deletions and insertions (input)}
\Require{$R^{t-1}$: Previous rank vector}
\Ensure{$R$: Current rank vector}
\Ensure{$\Delta r$: Change in rank of a vertex}
\Ensure{$\Delta R$: $L\infty$-norm between previous and current ranks}
\Ensure{$\alpha$: Damping factor}
\Ensure{$\tau$: Iteration tolerance}
\Ensure{$\tau_f$: Frontier tolerance}

\Statex

\Function{dynamicFrontier}{$G^{t-1}, G^t, \Delta^{t-}, \Delta^{t+}, R^{t-1}$}
  \State $R \gets R^{t-1}$
  \State $\rhd$ Mark initial affected
  \ForAll{$(u, v) \in \Delta^{t-} \cup \Delta^{t+} \textbf{in parallel}$} \label{alg:frontier--mark-begin}
    \ForAll{$v' \in (G^{t-1} \cup G^t).out(u)$}
    \State Mark $v'$ as affected
    \EndFor
  \EndFor \label{alg:frontier--mark-end}
  \ForAll{$i \in [0 .. MAX\_ITERATIONS)$} \label{alg:frontier--compute-begin}
    \State $\Delta R \gets 0$
    \ForAll{affected $v \in V^t$ \textbf{in parallel}}
      \State $r \gets (1 - \alpha)/|V^t|$
      \ForAll{$u \in G^t.in(v)$}
        \State $r \gets r + \alpha * R[u] / |G^t.out(u)|$
      \EndFor
      \State $\Delta r \gets |r - R[v]|$ \textbf{;} $\Delta R \gets max(\Delta R, \Delta r)$
      \State $\rhd$ Is rank change $>$ frontier tolerance?
      \If{$\Delta r > \tau_f$} \label{alg:frontier--remark-begin}
        \ForAll{$v' \in G^t.out(v)$}
          \State Mark $v'$ as affected
        \EndFor
      \EndIf \label{alg:frontier--remark-end}
    \EndFor
    \State $\rhd$ Ranks converged?
    \If{$\Delta R \le \tau$} \textbf{break}
    \EndIf
  \EndFor \label{alg:frontier--compute-end}
  \State \ReturnInline{$R$} \label{alg:frontier--return}
\EndFunction
\end{algorithmic}
\end{algorithm}




%% Requires (parameters):
% G(V, E): a directed unweighted graph
% R: initial ranks (1/N for static)

%% Parameter values:
% MAX\_ITERATIONS = 500
% DAMPING\_FACTOR = 0.85
% TOLERANCE = 10^-10





\subsection{Our Dynamic Frontier PageRank implementation}

Algorithm \ref{alg:frontier} shows our implementation of Dynamic Frontier PageRank, which is designed to compute the PageRank of vertices in a graph while efficiently handling dynamic changes in the graph structure over time. The algorithm takes as input the previous and current versions of the graph, edge deletions and insertions in the batch update, and the previous rank vector.

It begins by marking the initially affected vertices based on the edge deletions $\Delta^{t-}$ and insertions $\Delta^{t+}$ in parallel (lines \ref{alg:frontier--mark-begin}-\ref{alg:frontier--mark-end}). It then enters an iterative computation phase (lines \ref{alg:frontier--compute-begin}-\ref{alg:frontier--compute-end}), where it updates the rank of each affected vertex. The PageRank computation is performed in parallel for each affected vertex $v$, considering the incoming edges $G^t.in(v)$. The algorithm checks whether the change in rank $\Delta r$ exceeds the frontier tolerance $\tau_f$, and marks its out-neighbor vertices as affected if so. The iteration continues until either the net change in ranks $\Delta R$ (which is equal to the $L\infty$-norm between the previous and the current ranks) falls below the iteration tolerance $\tau$, or a maximum number of iterations is reached $MAX\_ITERATIONS$. In line \ref{alg:frontier--return}, the final rank vector $R$ is returned.




% Dynamic Frontier (DF) approach
% Adjusting tolerance, Frontier tolerance, Mark DelRank / DelContrib
% Dynamic Frontier optimizations
% Edge-balanced approach (Chunk size)


\section{Evaluation}
\label{sec:evaluation}
\subsection{Experimental Setup}
\label{sec:setup}

\subsubsection{System used}

Experiments are performed on a system featuring an AMD EPYC-7742 processor with $64$ cores, operating at a frequency of $2.25$ GHz. Each core is equipped with a $4$ MB L1 cache, a $32$ MB L2 cache, and shares a $256$ MB L3 cache. The server is set up with $512$ GB of DDR4 system memory and runs on Ubuntu $20.04$.


\subsubsection{Configuration}

We utilize 32-bit integers for vertex IDs and 64-bit floating-point numbers for vertex ranks. Affected vertices are represented using an 8-bit integer vector. The rank computation employs OpenMP's \textit{dynamic schedule} with a chunk size of $2048$, facilitating dynamic workload balancing among threads. We use a damping factor of $\alpha = 0.85$ \cite{rank-langville06}, with an iteration tolerance of $\tau = 10^{-10}$ using the $L_\infty$-norm \cite{rank-dubey22, rank-plimpton11}. The maximum number of iterations (\texttt{MAX\_ITERATIONS}) is limited to $500$ \cite{nvgraph}. All experiments are conducted with $64$ threads to match the number of cores available on the system, unless specified otherwise. Compilation is carried out using GCC $9.4$ and OpenMP $5.0$.


\subsubsection{Dataset}

We use the largest five temporal networks from the Stanford Large Network Dataset Collection \cite{snapnets}, as detailed in Table \ref{tab:dataset}. The number of vertices in these graphs range from $X$ million to $X$ million, with temporal edge counts spanning from $X$ million to $X$ billion, and static edge counts spanning from $X$ million to $X$ billion. For the experiment in Section $X$, we use four classes of static graphs (as batch update are randomly generated, with uniform probability for selection of any vertex as the endpoint of an edge to be inserted or deleted), sourced from the \textit{SuiteSparse Matrix Collection} \cite{suite19}, as detailed in Table \ref{tab:dataset}. The number of vertices in these graphs range from $3.07$ million to $214$ million, with edge counts spanning from $37.4$ million to $1.98$ billion. To address the impact of dead ends (vertices lacking out-links), a global teleport rank computation is needed in each iteration. We mitigate this overhead by adding self-loops to all vertices in the graph \cite{rank-andersen07, rank-langville06}.

\begin{table}[hbtp]
  \centering
  \caption{List of 5 real-world dynamic graphs\ignore{, i.e., temporal networks}, obtained from the Stanford Large Network Dataset Collection \cite{snapnets}. Here, $|V|$ is the number of vertices, $|E_T|$ the number of temporal edges\ignore{(includes duplicate edges)}, and $|E|$ the number of static edges (with no duplicates).\ignore{, and $\Gamma_G$ is the Gini coefficient of PageRank distribution. In the table, B refers to a billion, M refers to a million and K refers a thousand.}}
  \label{tab:dataset}
  \begin{tabular}{|c||c|c|c|c|}
    \toprule
    \textbf{Graph} &
    \textbf{\textbf{$|V|$}} &
    \textbf{\textbf{$|E_T|$}} &
    \textbf{\textbf{$|E|$}} \\
    \midrule
    sx-mathoverflow & 24.8K & 507K & 240K \\ \hline
    sx-askubuntu & 159K & 964K & 597K \\ \hline
    sx-superuser & 194K & 1.44M & 925K \\ \hline
    wiki-talk-temporal & 1.14M & 7.83M & 3.31M \\ \hline
    sx-stackoverflow & 2.60M & 63.4M & 36.2M \\ \hline
  \bottomrule
  \end{tabular}
\end{table}



\subsubsection{Batch Generation}
\label{sec:batch-generation}

For each base (static) graph from the dataset, we generate a random batch update, consisting of purely edge insertions, purely edge deletions, or an $80\% : 20\%$ mix of edge insertions and deletions to mimic realistic batch updates. The set of edges for insertion is prepared by selecting vertex pairs with equal probability. To construct the set of edge deletions, we delete each existing edge with a uniform probability. For simplicity, we ensure that no new vertices are added to or removed from the graph. The batch size is measured as a fraction of edges in the original graph, and is varied from $10^{-7}$ to $0.1$ (i.e., $10^{-7}|E|$ to $0.1|E|$), with multiple batches generated for each size (for averaging). Along with each batch update, self-loops are added to all vertices.


\subsubsection{Measurement}
\label{sec:measurement}

We measure the time taken by each approach on the updated graph entirely, including any preprocessing costs and convergence detection time, while excluding time dedicated to memory allocation and deallocation. The mean time for a specific method at a given batch size is calculated as the geometric mean across various input graphs. Consequently, the average speedup is determined as the ratio of these mean times. Additionally, we gauge the error/accuracy of a given approach by assessing the $L1$-norm \cite{ohsaka2015efficient} of the ranks in comparison to ranks obtained from a reference Static PageRank run on the updated graph with an extremely low iteration tolerance of $\tau = 10^{-100}$ (limited to $500$ iterations).

\begin{figure*}[!hbt]
  \centering
  \subfigure[]{
    \label{fig:temporal-all--runtime}
    \includegraphics[width=0.48\linewidth]{out/temporal-all-runtime.pdf}
  }
  \subfigure{
    \label{fig:temporal-all--error}
    \includegraphics[width=0.48\linewidth]{out/temporal-all-error.pdf}
  } \\[-4ex]
  \caption{Overall Runtime and Error in ranks obtained with \textit{Static}, \textit{Naive-dynamic (ND)}, \textit{Dynamic Traversal (DT)}, and our improved \textit{Dynamic Frontier (DF)} PageRank on real world dynamic graphs, with batch updates of size $10^{-5}|E|$ to $10^{-3}|E|$. In (a), the speedup of each approach with respect to Static PageRank is labeled.}
  \label{fig:temporal-all}
\end{figure*}

\input{src/fig-temporal-batch5}
\input{src/fig-temporal-batch4}
\input{src/fig-temporal-batch3}
\begin{figure*}[!hbt]
  \centering
  \subfigure{
    \label{fig:temporal-large--runtime}
    \includegraphics[width=0.48\linewidth]{out/temporal-large-runtime-sx-mathoverflow.pdf}
  }
  \subfigure{
    \label{fig:temporal-large--error}
    \includegraphics[width=0.48\linewidth]{out/temporal-large-error-sx-mathoverflow.pdf}
  }
  \subfigure{
    \label{fig:temporal-large--runtime}
    \includegraphics[width=0.48\linewidth]{out/temporal-large-runtime-sx-askubuntu.pdf}
  }
  \subfigure{
    \label{fig:temporal-large--error}
    \includegraphics[width=0.48\linewidth]{out/temporal-large-error-sx-askubuntu.pdf}
  }
  \subfigure{
    \label{fig:temporal-large--runtime}
    \includegraphics[width=0.48\linewidth]{out/temporal-large-runtime-sx-superuser.pdf}
  }
  \subfigure{
    \label{fig:temporal-large--error}
    \includegraphics[width=0.48\linewidth]{out/temporal-large-error-sx-superuser.pdf}
  }
  \subfigure{
    \label{fig:temporal-large--runtime}
    \includegraphics[width=0.48\linewidth]{out/temporal-large-runtime-wiki-talk-temporal.pdf}
  }
  \subfigure{
    \label{fig:temporal-large--error}
    \includegraphics[width=0.48\linewidth]{out/temporal-large-error-wiki-talk-temporal.pdf}
  }
  \subfigure{
    \label{fig:temporal-large--runtime}
    \includegraphics[width=0.48\linewidth]{out/temporal-large-runtime-sx-stackoverflow.pdf}
  }
  \subfigure{
    \label{fig:temporal-large--error}
    \includegraphics[width=0.48\linewidth]{out/temporal-large-error-sx-stackoverflow.pdf}
  } \\[-4ex]
  \caption{Overall Runtime and Error in ranks obtained with \textit{Static}, \textit{Naive-dynamic (ND)}, \textit{Dynamic Traversal (DT)}, and our improved \textit{Dynamic Frontier (DF)} PageRank on $100$ multi-batch updates from real world dynamic graphs, with batch updates of size $10^{-5}|E|$ to $10^{-3}|E|$. In each case, first $90\%$ of the real world graph is loaded, and then batch updates are applied one after the other.}
  \label{fig:temporal-large}
\end{figure*}

% \begin{figure*}[hbtp]
  \centering
  \subfigure[Overall result]{
    \label{fig:8020-runtime--mean}
    \includegraphics[width=0.38\linewidth]{out/8020-runtime-mean.pdf}
  }
  \subfigure[Results on each graph]{
    \label{fig:8020-runtime--all}
    \includegraphics[width=0.58\linewidth]{out/8020-runtime-all.pdf}
  } \\[-1ex]
  \caption{Runtime (logarithmic scale) of \textit{Static}, \textit{Naive-dynamic (ND)}, \textit{Dynamic Traversal (DT)}, our improved \textit{Dynamic Frontier (DF)}, and \textit{Dynamic Frontier with Pruning (DF-P)} PageRank on large (static) graphs with generated random batch updates, on batch updates of size $10^{-7}|E|$ to $0.1|E|$ in multiples of $10$. The updates include $80\%$ edge insertions and $20\%$ edge deletions, simulating realistic changes upon a dynamic graph. The subfigure on the right illustrates the runtime of each approach for each graph in the dataset, while the subfigure of the left presents overall runtimes (using geometric mean for consistent scaling across graphs). In addition, the speedup of each approach, relative to Static PageRank, is labeled on respective lines.}
  \label{fig:8020-runtime}
\end{figure*}

% \input{src/fig-8020-speedup}
% \begin{figure*}[hbtp]
  \centering
  \subfigure[Overall result]{
    \label{fig:8020-error--mean}
    \includegraphics[width=0.38\linewidth]{out/8020-error-mean.pdf}
  }
  \subfigure[Results on each graph]{
    \label{fig:8020-error--all}
    \includegraphics[width=0.58\linewidth]{out/8020-error-all.pdf}
  } \\[-1ex]
  \caption{Error comparison of \textit{Static}, \textit{Naive-dynamic (ND)}, \textit{Dynamic Traversal (DT)}, our improved \textit{Dynamic Frontier (DF)}, and \textit{Dynamic Frontier with Pruning (DF-P)} PageRank on large (static) graphs with generated random batch updates, relative to a Reference Static PageRank (see Section \ref{sec:measurement}), using $L1$-norm. The size of batch updates range from $10^{-7} |E|$ to $0.1 |E|$ in multiples of $10$ (logarithmic scale), consisting of $80\%$ edge insertions and $20\%$ edge deletions to simulate realistic dynamic graph updates. The right subfigure depicts the error for each approach in relation to each graph, while the left subfigure showcases overall errors using geometric mean for consistent scaling across graphs.}
  \label{fig:8020-error}
\end{figure*}

% \begin{figure}[!hbt]
  \centering
  \subfigure{
    \label{fig:measure-affected--batch}
    \includegraphics[width=0.98\linewidth]{out/measure-affected-batch.pdf}
  } \\[-2ex]
  \caption{Mean percentage of vertices marked as affected by \textit{Dynamic Traversal (DT)}, our improved \textit{Dynamic Frontier (DF)}, and \textit{Dynamic Frontier with Pruning (DF-P)} PageRank, on real-world graphs, with batch updates of size $10^{-5}|E_T|$ to $10^{-3}|E_T|$ (in multiples of $10$). DF and DF-P PageRank mark affected vertices incrementally --- thus, we count any vertex ever marked as affected.}
  \label{fig:measure-affected}
\end{figure}





\subsection{Performance of Dynamic Frontier PageRank}

We first study the performance of Dynamic Frontier PageRank on batch updates of size $10^{-7}|E|$ to $0.1|E|$ (in multiples of $10$), consisting purely of edge insertions, and compare it with Static, Naive-dynamic, and Dynamic Traversal PageRank. As mentioned above, the edge insertions are generated uniformly at random. Figure \ref{fig:insertions-runtime} plots the runtime of Static, Naive-dynamic, Dynamic Traversal, and Dynamic Frontier PageRank; Figure \ref{fig:insertions-speedup} plots the speedup of Dynamic Frontier PageRank with respect to Static, Naive-dynamic, and Dynamic Traversal PageRank; and Figure \ref{fig:insertions-error} plots the error in ranks obtained with Static, Naive-dynamic, Dynamic Traversal, and Dynamic Frontier PageRank with respect to ranks obtained from a reference Static PageRank (see Section \ref{sec:measurement}). In a similar manner, Figures \ref{fig:deletions-runtime}, \ref{fig:deletions-speedup}, and \ref{fig:deletions-error} present the runtime, speedup, and rank errors of each approach on batch updates consisting purely of edge deletions. Finally, Figures \ref{fig:8020-runtime}, \ref{fig:8020-speedup}, and \ref{fig:8020-error} present the runtime, speedup, and error with each approach on batch updates consisting of an $80\%$ / $20\%$ mix of edge insertions and deletions, in order to simulate realistic batch updates.


\subsubsection{Results with insertions-only batch updates}

Dynamic Frontier PageRank is on average $8.3\times$, $2.7\times$, and $3.4\times$ faster than Static, Naive-dynamic, and Dynamic Traversal PageRank on insertions-only batch updates of size $10^{-7}|E|$ to $10^{-3}|E|$, while obtaining ranks of better accuracy/error than Static PageRank, and of similar accuracy/error as Naive-dynamic and Dynamic Traversal PageRank. On road networks, and protein k-mer graphs, Dynamic Frontier PageRank is significantly faster than its competitors (Naive-dynamic and Dynamic Traversal PageRank).


\subsubsection{Results with deletions-only batch updates}

On deletions-only batch updates of size $10^{-7}|E|$ to $10^{-3}|E|$, Dynamic Frontier PageRank is on average $7.4\times$, $3.1\times$, and $4.1\times$ faster than Static, Naive-dynamic, and Dynamic Traversal PageRank, while obtaining ranks of better accuracy/error than Static PageRank (for batch sizes less than $0.1|E|$), and of similar accuracy/error as Naive-dynamic and Dynamic Traversal PageRank. On \textit{indochina-2004}, \textit{webbase-2001}, road networks, and protein k-mer graphs, Dynamic Frontier PageRank is significantly faster than its competitors (Naive-dynamic and Dynamic Traversal PageRank).


\subsubsection{Results with 80\%-20\% mix batch updates}

On batch updates of size $10^{-7}|E|$ to $10^{-3}|E|$, consisting of $80\%$ insertions and $20\%$ deletions, Dynamic Frontier PageRank is on average $7.6\times$, $2.8\times$, and $4.1\times$ faster than Static, Naive-dynamic, and Dynamic Traversal PageRank, while obtaining ranks of better accuracy/error than Static PageRank, and of similar accuracy/error as Naive-dynamic and Dynamic Traversal PageRank. Similar to deletions-only batch updates, Dynamic Frontier PageRank outperforms its competitors (Naive-dynamic and Dynamic Traversal PageRank) on \textit{indochina-2004}, \textit{webbase-2001}, road networks, and protein k-mer graphs.
% This seems to be associated to sparsity of the graphs as Dynamic Frontier PageRank performing well on sparse graphs.


\subsubsection{Results with temporal graphs}

We also attempt Static, Naive-dynamic, Dynamic Traversal, and Dynamic Frontier PageRank on temporal graphs found in the Stanford Large Network Dataset Collection \cite{snap14}. On some temporal graphs, Dynamic Frontier PageRank does not outperform its competitors with a frontier tolerance of $\tau_f = \tau / 10^5$, where $\tau$ is the iteration tolerance. However, choosing a lower $\tau_f$ of $\tau / 10$ or $\tau / 100$ allows it achieve good performance. Thus, the choice of frontier tolerance $\tau_f$, possibly in addition to how the frontier of affected vertices is expanded, is dependent upon the nature of the batch update. We plan to explore this in the future.


\subsubsection{Comparison of vertices marked as affected}

Figure \ref{fig:measure-affected} shows the total number of vertices marked as affected (average) by Dynamic Traversal and Dynamic Frontier PageRank on batch updates of size $10^{-7}|E|$ to $0.1|E|$, consisting exclusively of edge insertions. The Dynamic Frontier approach marks affected vertices incrementally --- thus, the final percentage (at the end of all iterations) is depicted in the figure. It is observed that Dynamic Traversal PageRank marks a higher percentage of vertices as affected, even for small batch updates.\ignore{This is likely due the randomly generated edges in the batch update being part of large Strongly Connected Components (SCCs), or due to a large number of such SCCs being reachable from the vertices that are part of the batch update.} In contrast, Dynamic Frontier PageRank marks far fewer vertices as affected, as it incrementally expands the affected region of the graph only after the rank of an affected vertex changes by a substantial amount, i.e., by frontier tolerance $\tau_f = \tau / 10^5$, where $\tau$ is the iteration tolerance (using $L\infty$-norm). In addition, as Dynamic Frontier PageRank incrementally marks vertices as affected, the actual work performed by the algorithm is lower than that indicated by the percentage of affected vertices in Figure \ref{fig:measure-affected}.

\begin{figure}[!hbt]
  \centering
  \subfigure{
    \label{fig:strong-scaling--speedup}
    \includegraphics[width=0.98\linewidth]{out/strong-scaling-speedup.pdf}
  } \\[-2ex]
  \caption{Mean speedup of our improved \textit{Dynamic Frontier (DF)} and \textit{Dynamic Frontier with Pruning (DF-P)} PageRank with increasing number of threads (in multiples of $2$), on real-world dynamic graphs, with batch updates of size $10^{-4}|E_T|$.}
  \label{fig:strong-scaling}
\end{figure}





\subsection{Strong Scaling of Dynamic Frontier PageRank}

Finally, we study the strong-scaling behavior of Dynamic Frontier PageRank on batch updates of a fixed size of $10^{-4} |E|$, consisting purely of edge insertions. Here, we measure the speedup of Dynamic Frontier PageRank with an increasing number of threads from $1$ to $64$ in multiples of $2$ with respect to a single-threaded execution of the algorithm. This is repeated for each graph in the dataset, and the results are averaged (using geometric mean).

The results are shown in Figure \ref{fig:strong-scaling}. With $16$ threads, Dynamic Frontier PageRank achieves an average speedup of $10.3\times$, compared to a single-threaded execution, indicating a performance increase of $1.8\times$ for every doubling of threads. At $32$ and $64$ threads, Dynamic Frontier PageRank is affected by NUMA effects (the $64$-core processor we use has $4$ NUMA domains), resulting in a speedup of only $14.3\times$ and $15.2\times$ respectively.

\ignore{\begin{figure}[!hbt]
  \centering
  \subfigure{
    \label{fig:weak-scaling--speedup}
    \includegraphics[width=0.98\linewidth]{out/weak-scaling-speedup.pdf}
  } \\[-2ex]
  \caption{Average speedup of \textit{Dynamic Frontier} PageRank with increasing number of threads (in multiples of $2$), on a batch sizes of $10^{-4}|E|$ to $6.4\times10^{-3}|E|$ (consisting purely of edge insertions), increasing in multiples of $2$ in tandem with the increase in the number of threads.}
  \label{fig:weak-scaling}
\end{figure}
}


\section{Conclusion}
\label{sec:conclusion}
In conclusion, this study presents an efficient algorithm for updating PageRank on dynamic graphs. Given a batch update of edge insertions and deletions, our Dynamic Frontier approach identifies an initial set of affected vertices and incrementally expands this set through iterations. On a server with a 64-core AMD EPYC-7742 processor, Dynamic Frontier PageRank outperforms Static, Naive-dynamic, and Dynamic Traversal PageRank by $8.3\times$, $2.7\times$, and $3.4\times$ respectively for uniformly random batch updates of size $10^{-7}|E|$ to $10^{-3}|E|$ with purely edge insertions; $7.4\times$, $3.1\times$, and $4.1\times$ respectively for purely edge deletion updates; and $7.6\times$, $2.8\times$, and $4.1\times$ for updates consisting of an $80\%$ - $20\%$ mix of insertions and deletions. Additionally, the approach exhibits a performance improvement of $1.8\times$ for each doubling of threads. On temporal graphs, we observe that lowering $\tau_f$ to $\tau / 10$ or $\tau / 100$ is needed for Dynamic Frontier PageRank to achieve food performance. Thus, a suitable choice of $\tau_f$ and how the frontier of affected vertices expands depend on the batch update's nature. We plan to explore this in the future.


%% The acknowledgments section.
\begin{acks}
I would like to thank Prof. Kishore Kothapalli, Prof. Sathya Peri, and Prof. Hemalatha Eedi for their support.\ignore{Note that Britannia Industries Ltd., the owner of the 50-50 biscuit brand, did not sponsor our work.}
\end{acks}

%% Bibliography style to be used, and the bibliography file.
\bibliographystyle{ACM-Reference-Format}
\bibliography{main}

\clearpage
\appendix
% \section{Appendix}
\begin{figure*}[!hbt]
  \centering
  \subfigure[Runtime on consecutive batch updates of size $10^{-5}|E_T|$]{
    \label{fig:temporal-sx-mathoverflow--runtime5}
    \includegraphics[width=0.48\linewidth]{out/temporal-sx-mathoverflow-runtime5.pdf}
  }
  \subfigure[Error in ranks obtained on consecutive batch updates of size $10^{-5}|E_T|$]{
    \label{fig:temporal-sx-mathoverflow--error5}
    \includegraphics[width=0.48\linewidth]{out/temporal-sx-mathoverflow-error5.pdf}
  } \\[2ex]
  \subfigure[Runtime on consecutive batch updates of size $10^{-4}|E_T|$]{
    \label{fig:temporal-sx-mathoverflow--runtime4}
    \includegraphics[width=0.48\linewidth]{out/temporal-sx-mathoverflow-runtime4.pdf}
  }
  \subfigure[Error in ranks obtained on consecutive batch updates of size $10^{-4}|E_T|$]{
    \label{fig:temporal-sx-mathoverflow--error4}
    \includegraphics[width=0.48\linewidth]{out/temporal-sx-mathoverflow-error4.pdf}
  } \\[2ex]
  \subfigure[Runtime on consecutive batch updates of size $10^{-3}|E_T|$]{
    \label{fig:temporal-sx-mathoverflow--runtime3}
    \includegraphics[width=0.48\linewidth]{out/temporal-sx-mathoverflow-runtime3.pdf}
  }
  \subfigure[Error in ranks obtained on consecutive batch updates of size $10^{-3}|E_T|$]{
    \label{fig:temporal-sx-mathoverflow--error3}
    \includegraphics[width=0.48\linewidth]{out/temporal-sx-mathoverflow-error3.pdf}
  } \\[-2ex]
  \caption{Runtime and Error in ranks obtained with \textit{Static}, \textit{Naive-dynamic (ND)}, \textit{Dynamic Traversal (DT)}, our improved \textit{Dynamic Frontier (DF)}, and our improved \textit{Dynamic Frontier with Pruning (DF-P)} PageRank on the \textit{sx-mathoverflow} dynamic graph. The size of batch updates range from $10^{-5}|E_T|$ to $10^{-3}|E_T|$. The rank error with each approach is measured relative to ranks obtained with a reference Static PageRank run, as detailed in Section \ref{sec:measurement}.}
  \label{fig:temporal-sx-mathoverflow}
\end{figure*}

\input{src/fig-temporal-sx-askubuntu}
\begin{figure*}[!hbt]
  \centering
  \subfigure[Runtime on consecutive batch updates of size $10^{-5}|E_T|$]{
    \label{fig:temporal-sx-superuser--runtime5}
    \includegraphics[width=0.48\linewidth]{out/temporal-sx-superuser-runtime5.pdf}
  }
  \subfigure[Error in ranks obtained on consecutive batch updates of size $10^{-5}|E_T|$]{
    \label{fig:temporal-sx-superuser--error5}
    \includegraphics[width=0.48\linewidth]{out/temporal-sx-superuser-error5.pdf}
  } \\[2ex]
  \subfigure[Runtime on consecutive batch updates of size $10^{-4}|E_T|$]{
    \label{fig:temporal-sx-superuser--runtime4}
    \includegraphics[width=0.48\linewidth]{out/temporal-sx-superuser-runtime4.pdf}
  }
  \subfigure[Error in ranks obtained on consecutive batch updates of size $10^{-4}|E_T|$]{
    \label{fig:temporal-sx-superuser--error4}
    \includegraphics[width=0.48\linewidth]{out/temporal-sx-superuser-error4.pdf}
  } \\[2ex]
  \subfigure[Runtime on consecutive batch updates of size $10^{-3}|E_T|$]{
    \label{fig:temporal-sx-superuser--runtime3}
    \includegraphics[width=0.48\linewidth]{out/temporal-sx-superuser-runtime3.pdf}
  }
  \subfigure[Error in ranks obtained on consecutive batch updates of size $10^{-3}|E_T|$]{
    \label{fig:temporal-sx-superuser--error3}
    \includegraphics[width=0.48\linewidth]{out/temporal-sx-superuser-error3.pdf}
  } \\[-2ex]
  \caption{Runtime and Error in ranks obtained with \textit{Static}, \textit{Naive-dynamic (ND)}, \textit{Dynamic Traversal (DT)}, our improved \textit{Dynamic Frontier (DF)}, and our improved \textit{Dynamic Frontier with Pruning (DF-P)} PageRank on the \textit{sx-superuser} dynamic graph. The size of batch updates range from $10^{-5}|E_T|$ to $10^{-3}|E_T|$. The rank error with each approach is measured relative to ranks obtained with a reference Static PageRank run, as detailed in Section \ref{sec:measurement}.}
  \label{fig:temporal-sx-superuser}
\end{figure*}

\begin{figure*}[!hbt]
  \centering
  \subfigure[Runtime on consecutive batch updates of size $10^{-5}|E_T|$]{
    \label{fig:temporal-wiki-talk-temporal--runtime5}
    \includegraphics[width=0.48\linewidth]{out/temporal-wiki-talk-temporal-runtime5.pdf}
  }
  \subfigure[Error in ranks obtained on consecutive batch updates of size $10^{-5}|E_T|$]{
    \label{fig:temporal-wiki-talk-temporal--error5}
    \includegraphics[width=0.48\linewidth]{out/temporal-wiki-talk-temporal-error5.pdf}
  } \\[2ex]
  \subfigure[Runtime on consecutive batch updates of size $10^{-4}|E_T|$]{
    \label{fig:temporal-wiki-talk-temporal--runtime4}
    \includegraphics[width=0.48\linewidth]{out/temporal-wiki-talk-temporal-runtime4.pdf}
  }
  \subfigure[Error in ranks obtained on consecutive batch updates of size $10^{-4}|E_T|$]{
    \label{fig:temporal-wiki-talk-temporal--error4}
    \includegraphics[width=0.48\linewidth]{out/temporal-wiki-talk-temporal-error4.pdf}
  } \\[2ex]
  \subfigure[Runtime on consecutive batch updates of size $10^{-3}|E_T|$]{
    \label{fig:temporal-wiki-talk-temporal--runtime3}
    \includegraphics[width=0.48\linewidth]{out/temporal-wiki-talk-temporal-runtime3.pdf}
  }
  \subfigure[Error in ranks obtained on consecutive batch updates of size $10^{-3}|E_T|$]{
    \label{fig:temporal-wiki-talk-temporal--error3}
    \includegraphics[width=0.48\linewidth]{out/temporal-wiki-talk-temporal-error3.pdf}
  } \\[-2ex]
  \caption{Runtime and Error in ranks obtained with \textit{Static}, \textit{Naive-dynamic (ND)}, \textit{Dynamic Traversal (DT)}, our improved \textit{Dynamic Frontier (DF)}, and our improved \textit{Dynamic Frontier with Pruning (DF-P)} PageRank on the \textit{wiki-talk-temporal} dynamic graph. The size of batch updates range from $10^{-5}|E_T|$ to $10^{-3}|E_T|$. The rank error with each approach is measured relative to ranks obtained with a reference Static PageRank run, as detailed in Section \ref{sec:measurement}.}
  \label{fig:temporal-wiki-talk-temporal}
\end{figure*}

\begin{figure*}[!hbt]
  \centering
  \subfigure[Runtime on consecutive batch updates of size $10^{-5}|E_T|$]{
    \label{fig:temporal-sx-stackoverflow--runtime5}
    \includegraphics[width=0.48\linewidth]{out/temporal-sx-stackoverflow-runtime5.pdf}
  }
  \subfigure[Error in ranks obtained on consecutive batch updates of size $10^{-5}|E_T|$]{
    \label{fig:temporal-sx-stackoverflow--error5}
    \includegraphics[width=0.48\linewidth]{out/temporal-sx-stackoverflow-error5.pdf}
  } \\[2ex]
  \subfigure[Runtime on consecutive batch updates of size $10^{-4}|E_T|$]{
    \label{fig:temporal-sx-stackoverflow--runtime4}
    \includegraphics[width=0.48\linewidth]{out/temporal-sx-stackoverflow-runtime4.pdf}
  }
  \subfigure[Error in ranks obtained on consecutive batch updates of size $10^{-4}|E_T|$]{
    \label{fig:temporal-sx-stackoverflow--error4}
    \includegraphics[width=0.48\linewidth]{out/temporal-sx-stackoverflow-error4.pdf}
  } \\[2ex]
  \subfigure[Runtime on consecutive batch updates of size $10^{-3}|E_T|$]{
    \label{fig:temporal-sx-stackoverflow--runtime3}
    \includegraphics[width=0.48\linewidth]{out/temporal-sx-stackoverflow-runtime3.pdf}
  }
  \subfigure[Error in ranks obtained on consecutive batch updates of size $10^{-3}|E_T|$]{
    \label{fig:temporal-sx-stackoverflow--error3}
    \includegraphics[width=0.48\linewidth]{out/temporal-sx-stackoverflow-error3.pdf}
  } \\[-2ex]
  \caption{Runtime and Error in ranks obtained with \textit{Static}, \textit{Naive-dynamic (ND)}, \textit{Dynamic Traversal (DT)}, our improved \textit{Dynamic Frontier (DF)}, and our improved \textit{Dynamic Frontier with Pruning (DF-P)} PageRank on the \textit{sx-stackoverflow} dynamic graph. The size of batch updates range from $10^{-5}|E_T|$ to $10^{-3}|E_T|$. The rank error with each approach is measured relative to ranks obtained with a reference Static PageRank run, as detailed in Section \ref{sec:measurement}.}
  \label{fig:temporal-sx-stackoverflow}
\end{figure*}

\begin{figure*}[hbtp]
  \centering
  \subfigure[Overall result]{
    \label{fig:8020-runtime--mean}
    \includegraphics[width=0.38\linewidth]{out/8020-runtime-mean.pdf}
  }
  \subfigure[Results on each graph]{
    \label{fig:8020-runtime--all}
    \includegraphics[width=0.58\linewidth]{out/8020-runtime-all.pdf}
  } \\[-1ex]
  \caption{Runtime (logarithmic scale) of \textit{Static}, \textit{Naive-dynamic (ND)}, \textit{Dynamic Traversal (DT)}, our improved \textit{Dynamic Frontier (DF)}, and \textit{Dynamic Frontier with Pruning (DF-P)} PageRank on large (static) graphs with generated random batch updates, on batch updates of size $10^{-7}|E|$ to $0.1|E|$ in multiples of $10$. The updates include $80\%$ edge insertions and $20\%$ edge deletions, simulating realistic changes upon a dynamic graph. The subfigure on the right illustrates the runtime of each approach for each graph in the dataset, while the subfigure of the left presents overall runtimes (using geometric mean for consistent scaling across graphs). In addition, the speedup of each approach, relative to Static PageRank, is labeled on respective lines.}
  \label{fig:8020-runtime}
\end{figure*}

\begin{figure*}[hbtp]
  \centering
  \subfigure[Overall result]{
    \label{fig:8020-error--mean}
    \includegraphics[width=0.38\linewidth]{out/8020-error-mean.pdf}
  }
  \subfigure[Results on each graph]{
    \label{fig:8020-error--all}
    \includegraphics[width=0.58\linewidth]{out/8020-error-all.pdf}
  } \\[-1ex]
  \caption{Error comparison of \textit{Static}, \textit{Naive-dynamic (ND)}, \textit{Dynamic Traversal (DT)}, our improved \textit{Dynamic Frontier (DF)}, and \textit{Dynamic Frontier with Pruning (DF-P)} PageRank on large (static) graphs with generated random batch updates, relative to a Reference Static PageRank (see Section \ref{sec:measurement}), using $L1$-norm. The size of batch updates range from $10^{-7} |E|$ to $0.1 |E|$ in multiples of $10$ (logarithmic scale), consisting of $80\%$ edge insertions and $20\%$ edge deletions to simulate realistic dynamic graph updates. The right subfigure depicts the error for each approach in relation to each graph, while the left subfigure showcases overall errors using geometric mean for consistent scaling across graphs.}
  \label{fig:8020-error}
\end{figure*}


\clearpage

\section{Appendix}

\subsection{Derivation of Closed loop formula for Rank calculation towards Dynamic Frontier with Pruning (DF-P) PageRank}
\label{sec:pr-prune-derivation}

We proceed to derive the closed-loop formula for rank calculation with DF-P PageRank. As outlined in Sections \ref{sec:dataset} and \ref{sec:batch-generation}, self-loops are added to each vertex to circumvent the need for a global teleport rank computation in every iteration, thus reducing overhead. In DF-P PageRank, our aim is to skip the computation of ranks for vertices likely to have already converged. However, the existence of self-loops causes a delay in vertex rank convergence due to the immediate recursive nature they introduce. For instance, if the ranks of all in-neighbors of a vertex have already converged, the presence of self-loops inhibits the convergence of the vertex's rank in a single iteration. Nevertheless, we can mitigate this convergence issue by employing a closed-loop formula for the rank calculation of each vertex.

To achieve this, let us denote $r_0$ as the initial rank of a vertex $v$, $\alpha$ as the damping factor, $c = \sum_{u \in G.in(v)} \frac{R[u]}{|G.out(u)|}$ as the total rank contribution from its in-neighbors (excluding itself), $d = |G.out(v)|$ as its out-degree, and $C_0$ as $1 - \alpha/|V|$. Given the assumption that the rank contribution of its in-neighbors remains constant, the rank of $v$ after one iteration can be expressed as:

\begin{flalign*}
  r_1 & = \alpha (c + \frac{r_0}{d}) + C_0 && \\
      & = \alpha c + \alpha \frac{r_0}{d} + C_0 && \\
\end{flalign*}

\noindent
After the second iteration, the rank of the vertex would be:

\begin{flalign*}
  r_2 & = \alpha (c + \frac{r_1}{d}) + C_0 && \\
      & = \alpha (c + \frac{1}{d} (\alpha c + \alpha \frac{r_0}{d} + C_0)) + C_0 && \\
      & = \alpha c + \alpha^2 \frac{c}{d} + \alpha^2 \frac{r_0}{d^2} + \alpha \frac{C_0}{d} + C_0 &&
\end{flalign*}

\noindent
Following the third iteration, the vertex's rank would be:

\begin{flalign*}
  r_3 & = \alpha (c + \frac{r_2}{d}) + C_0 && \\
      & = \alpha (c + \frac{1}{d} (\alpha c + \alpha^2 \frac{c}{d} + \alpha^2 \frac{r_0}{d^2} + \alpha \frac{C_0}{d} + C_0) + C_0 && \\
      & = \alpha c + \alpha^2 \frac{c}{d} + \alpha^3 \frac{c}{d^2} + \alpha^3 \frac{r_0}{d^3} + \alpha^2 \frac{C_0}{d^2} + \alpha \frac{C_0}{d} + C_0 && \\
\end{flalign*}

\noindent
Expanding this to an infinite number of iterations, the vertex's final rank would be:

\begin{flalign*}
  r_\infty & = \frac{\alpha c}{1 - \alpha / d} + \frac{C_0}{1 - \alpha / d} && \\
           & = \frac{1}{1 - \alpha / d} (\alpha c + C_0)
\end{flalign*}

\noindent
Hence, the closed-loop formula for calculating the rank of a vertex $v$ in DF-P PageRank is:

\begin{flalign}
  R[v] & = \frac{1}{1 - \alpha / |G.out(v)|} \left(\alpha K + \frac{1 - \alpha}{|V|}\right) && \\
    \text{where, } K & = \left(\sum_{u \in G.in(v)} \frac{R[u]}{|G.out(u)|}\right) - \frac{R[v]}{|G.out(v)|}
\end{flalign}

\begin{table}[hbtp]
  \centering
  \caption{List of $12$ graphs sourced from the SuiteSparse Matrix Collection \cite{suite19}, where directed graphs are indicated with $*$. Here, $|V|$ denotes the number of vertices, $|E|$ represents the number of edges (inclusive of self-loops), and $D_{avg}$ represents the average degree.}
  \label{tab:dataset-large}
  \begin{tabular}{|c||c|c|c|c|}
    \toprule
    \textbf{Graph} &
    \textbf{\textbf{$|V|$}} &
    \textbf{\textbf{$|E|$}} &
    \textbf{\textbf{$D_{avg}$}} \\
    \midrule
    \multicolumn{4}{|c|}{\textbf{Web Graphs (LAW)}} \\ \hline
    indochina-2004$^*$ & 7.41M & 199M & 26.8 \\ \hline  % & \num{4.7e-4}
    % uk-2002$^*$ & 18.5M & 311M & 16.8 \\ \hline  % & \num{9.6e-5}
    arabic-2005$^*$ & 22.7M & 654M & 28.8 \\ \hline  % & \num{5.5e-4}
    uk-2005$^*$ & 39.5M & 961M & 24.3 \\ \hline  % & \num{9.6e-5}
    webbase-2001$^*$ & 118M & 1.11B & 9.4 \\ \hline  % & \num{7.3e-7}
    it-2004$^*$ & 41.3M & 1.18B & 28.5 \\ \hline  % & \num{3.8e-4}
    sk-2005$^*$ & 50.6M & 1.98B & 39.1 \\ \hline  % & \num{5.8e-4}
    \multicolumn{4}{|c|}{\textbf{Social Networks (SNAP)}} \\ \hline
    com-LiveJournal & 4.00M & 73.4M & 18.3 \\ \hline  % & \num{7.9e-4}
    com-Orkut & 3.07M & 237M & 77.3 \\ \hline  % & \num{6.7e-2}
    \multicolumn{4}{|c|}{\textbf{Road Networks (DIMACS10)}} \\ \hline
    asia\_osm & 12.0M & 37.4M & 3.1 \\ \hline  % & \num{8.4e-4}
    europe\_osm & 50.9M & 159M & 3.1 \\ \hline  % & \num{6.6e-4}
    \multicolumn{4}{|c|}{\textbf{Protein k-mer Graphs (GenBank)}} \\ \hline
    kmer\_A2a & 171M & 531M & 3.1 \\ \hline  % & \num{9.4e-5}
    kmer\_V1r & 214M & 679M & 3.2 \\ \hline  % & \num{3.2e-4}
  \bottomrule
  \end{tabular}
\end{table}


\end{document}
\endinput
%% End of file.
