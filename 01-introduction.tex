Centrality is the problem of quantifying the significance of nodes within a network based on link structures. PageRank \cite{rank-page99}, originally proposed to rank web pages in search results, is one the most popular centrality metrics. It is based on the principle that pages receiving a greater number of high-quality links are of higher quality and, consequently, should be assigned higher ranks. Due to the importance of such a metric, its also finds applications in areas other than web page ranking, such as, urban planning \cite{urban-zhang18}, traffic flow prediction \cite{traffic-kim15}, and protein target identification \cite{banky2013equal}.

The escalating availability of graph-based data has spurred a considerable interest in parallel algorithms for PageRank computation \cite{rank-garg16, rank-nvgraph, rank-giri20, rank-sarma13}.\ignore{--- implementations span multicore CPUs \cite{rank-garg16}, GPUs \cite{rank-nvgraph}, FPGAs \cite{rank-guoqiang20}, SpMV ASICs \cite{rank-sadi18}, CPU-GPU hybrids \cite{rank-giri20}, CPU-FPGA hybrids \cite{rank-li21}, and distributed systems \cite{rank-sarma13}.}

%% FROM COMPRE
PageRank is used for word sense \textbf{disambiguation} in lexical semantics (PPR) \cite{de2010robust, duque2015co, agirre2010graph}, urban planning \cite{urban-zhang18}, ranking streets by traffic \cite{traffic-kim15}, identifying \textbf{communities} \cite{rank-kloumann17}, measuring their \textbf{impact} on the web, maximizing influence \cite{influence-zhang15}, providing \textbf{recommendations} (PPR) \cite{recommend-chaudhari17}, analyzing neural/protein networks, determining species \textbf{essential} for health of the environment \cite{allesina2009googling}, or even quantifying the \textbf{scientific impact} of researchers \cite{rank-senanayake15}.


%% RELATED WORK
PageRank is used in image search to identify canonical to display as a visual summary of a larger set of images returned from an image search engine \cite{rank-gleich15}.

Kolda and Procopio \cite{kolda2009generalized} propose generalized models (based on BadRank \cite{sobek2002pr0} and TrustRank \cite{rank-gyongyi04}) to combat spam sites that create no new information content, but attempt to capture Google search results by appearing to contain information, and include the idea of adding self-links to all nodes in the graph, to fix the dangling nodes. For spam-link applications, this method of handling dangling nodes is superior --- in a modeling sense --- to the alternatives.

Chepelianskii \cite{chepelianskii2010towards} uses the distribution of PageRank and reverse PageRank scores to characterize the properties of the linux kernel (a software system), using a call graph of the linux kernel.

Zuo et al. \cite{zuo2012network} use PageRank to evaluate the importance of brain regions given observed correlations of brain activity.

% PageRank \cite{rank-page99} is an algorithm that measures the importance of nodes in a network by assigning numerical scores based on the structure of links. It finds applications in web page ranking, identifying misinformation, predicting traffic flow, and protein target identification. The increasing availability of vast amounts of data represented as graphs has led to a significant interest in parallel algorithms for computing PageRank \cite{rank-garg16, rank-nvgraph, rank-giri20, rank-sarma13}.\ignore{--- it has been implemented on multicore CPUs \cite{rank-garg16}, GPUs \cite{rank-nvgraph}, FPGAs \cite{rank-guoqiang20}, SpMV ASICs \cite{rank-sadi18}, CPU-GPU hybrids \cite{rank-giri20}, CPU-FPGA hybrids \cite{rank-li21}, and distributed systems \cite{rank-sarma13}.}

However, most real-world graph evolve with time. Here, frequent edge insertions and deletions make recomputing PageRank from scratch impractical, particularly for small, rapid changes. Existing strategies optimize by iterating from the prior snapshot's ranks, reducing the number of iterations needed for convergence. For further improvements, it is essential to recompute only the ranks of vertices likely to change. A prevalent approach involves identifying reachable vertices from the updated regions of the graph, and limiting processing to such vertices. However, if updates are randomly distributed, they often fall within dense graph regions, necessitating processing of a substantial portion of the graph.

To reduce computational effort, one can incrementally expand the set of affected vertices starting from the updated graph region, rather than processing all reachable vertices from the first iteration. Additionally, it is possible to skip processing a vertex's neighbors if the change in its rank is small and is expected to have minimal impact on the ranks of its neighboring vertices. This technical report introduces such an approach.




\subsection{Our Contributions}

This report introduces our Dynamic Frontier approach\footnote{\url{https://github.com/puzzlef/pagerank-openmp-dynamic}}, which, when given a batch update involving edge insertions and deletions, incrementally identifies affected vertices likely to undergo rank changes with minimal overhead. On a server equipped with a 64-core AMD EPYC-7742 processor, our Dynamic Frontier PageRank surpasses Static, Naive-dynamic, and Dynamic Traversal PageRank by $7.8\times$, $2.9\times$, and $3.9\times$ respectively, for uniformly random batch updates of size $10^{-7}|E|$ to $10^{-3}|E|$, where $|E|$ is the number of edges in the original graph. Additionally, our approach exhibits a performance improvement of $1.8\times$ for each doubling of threads.




%% - Use --- for a dash.
%% - Use ``camera-ready'' for quotes.
%% - Use {\itshape very} or \textit{very} for italicized text.
%% - Use \verb|acmart| or {\verb|acmart|} for mono-spaced text.
%% - Use \url{https://capitalizemytitle.com/} for URLs.
%% - Use {\bfseries Do not modify this document.} for important boldface details.
%% - Use \ref{fig:name} for referencing.

%% For a block of pre-formatted text: 
% \begin{verbatim}
%   \renewcommand{\shortauthors}{McCartney, et al.}
% \end{verbatim}

%% For a list of items:
% \begin{itemize}
% \item the ``ACM Reference Format'' text on the first page.
% \item the ``rights management'' text on the first page.
% \item the conference information in the page header(s).
% \end{itemize}

%% For a table:
% \begin{table}
%   \caption{Frequency of Special Characters}
%   \label{tab:freq}
%   \begin{tabular}{ccl}
%     \toprule
%     Non-English or Math&Frequency&Comments\\
%     \midrule
%     \O & 1 in 1,000& For Swedish names\\
%     $\pi$ & 1 in 5& Common in math\\
%     \$ & 4 in 5 & Used in business\\
%     $\Psi^2_1$ & 1 in 40,000& Unexplained usage\\
%   \bottomrule
% \end{tabular}
% \end{table}

%% For a full-width table:
% \begin{table*}
%   \caption{Some Typical Commands}
%   \label{tab:commands}
%   \begin{tabular}{ccl}
%     \toprule
%     Command &A Number & Comments\\
%     \midrule
%     \texttt{{\char'134}author} & 100& Author \\
%     \texttt{{\char'134}table}& 300 & For tables\\
%     \texttt{{\char'134}table*}& 400& For wider tables\\
%     \bottomrule
%   \end{tabular}
% \end{table*}


%% For inline math:
% \begin{math}
%   \lim_{n\rightarrow \infty}x=0
% \end{math},

%% For a numbered equation:
% \begin{equation}
%   \lim_{n\rightarrow \infty}x=0
% \end{equation}

%% For an unnumbered equation:
% \begin{displaymath}
%   \sum_{i=0}^{\infty} x + 1
% \end{displaymath}

%% For a figure:
% \begin{figure}[h]
%   \centering
%   \includegraphics[width=\linewidth]{inc/sample-franklin}
%   \caption{1907 Franklin Model D roadster. Photograph by Harris \&
%     Ewing, Inc. [Public domain], via Wikimedia
%     Commons. (\url{https://goo.gl/VLCRBB}).}
%   \Description{A woman and a girl in white dresses sit in an open car.}
% \end{figure}

%% For a teaser figure.
% \begin{teaserfigure}
%   \includegraphics[width=\textwidth]{sampleteaser}
%   \caption{figure caption}
%   \Description{figure description}
% \end{teaserfigure}
