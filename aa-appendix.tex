\begin{figure*}[!hbt]
  \centering
  \subfigure[Runtime on consecutive batch updates of size $10^{-5}|E_T|$]{
    \label{fig:temporal-sx-mathoverflow--runtime5}
    \includegraphics[width=0.48\linewidth]{out/temporal-sx-mathoverflow-runtime5.pdf}
  }
  \subfigure[Error in ranks obtained on consecutive batch updates of size $10^{-5}|E_T|$]{
    \label{fig:temporal-sx-mathoverflow--error5}
    \includegraphics[width=0.48\linewidth]{out/temporal-sx-mathoverflow-error5.pdf}
  } \\[2ex]
  \subfigure[Runtime on consecutive batch updates of size $10^{-4}|E_T|$]{
    \label{fig:temporal-sx-mathoverflow--runtime4}
    \includegraphics[width=0.48\linewidth]{out/temporal-sx-mathoverflow-runtime4.pdf}
  }
  \subfigure[Error in ranks obtained on consecutive batch updates of size $10^{-4}|E_T|$]{
    \label{fig:temporal-sx-mathoverflow--error4}
    \includegraphics[width=0.48\linewidth]{out/temporal-sx-mathoverflow-error4.pdf}
  } \\[2ex]
  \subfigure[Runtime on consecutive batch updates of size $10^{-3}|E_T|$]{
    \label{fig:temporal-sx-mathoverflow--runtime3}
    \includegraphics[width=0.48\linewidth]{out/temporal-sx-mathoverflow-runtime3.pdf}
  }
  \subfigure[Error in ranks obtained on consecutive batch updates of size $10^{-3}|E_T|$]{
    \label{fig:temporal-sx-mathoverflow--error3}
    \includegraphics[width=0.48\linewidth]{out/temporal-sx-mathoverflow-error3.pdf}
  } \\[-2ex]
  \caption{Runtime and Error in ranks obtained with \textit{Static}, \textit{Naive-dynamic (ND)}, \textit{Dynamic Traversal (DT)}, our improved \textit{Dynamic Frontier (DF)}, and our improved \textit{Dynamic Frontier with Pruning (DF-P)} PageRank on the \textit{sx-mathoverflow} dynamic graph. The size of batch updates range from $10^{-5}|E_T|$ to $10^{-3}|E_T|$. The rank error with each approach is measured relative to ranks obtained with a reference Static PageRank run, as detailed in Section \ref{sec:measurement}.}
  \label{fig:temporal-sx-mathoverflow}
\end{figure*}

\input{src/fig-temporal-sx-askubuntu}
\begin{figure*}[!hbt]
  \centering
  \subfigure[Runtime on consecutive batch updates of size $10^{-5}|E_T|$]{
    \label{fig:temporal-sx-superuser--runtime5}
    \includegraphics[width=0.48\linewidth]{out/temporal-sx-superuser-runtime5.pdf}
  }
  \subfigure[Error in ranks obtained on consecutive batch updates of size $10^{-5}|E_T|$]{
    \label{fig:temporal-sx-superuser--error5}
    \includegraphics[width=0.48\linewidth]{out/temporal-sx-superuser-error5.pdf}
  } \\[2ex]
  \subfigure[Runtime on consecutive batch updates of size $10^{-4}|E_T|$]{
    \label{fig:temporal-sx-superuser--runtime4}
    \includegraphics[width=0.48\linewidth]{out/temporal-sx-superuser-runtime4.pdf}
  }
  \subfigure[Error in ranks obtained on consecutive batch updates of size $10^{-4}|E_T|$]{
    \label{fig:temporal-sx-superuser--error4}
    \includegraphics[width=0.48\linewidth]{out/temporal-sx-superuser-error4.pdf}
  } \\[2ex]
  \subfigure[Runtime on consecutive batch updates of size $10^{-3}|E_T|$]{
    \label{fig:temporal-sx-superuser--runtime3}
    \includegraphics[width=0.48\linewidth]{out/temporal-sx-superuser-runtime3.pdf}
  }
  \subfigure[Error in ranks obtained on consecutive batch updates of size $10^{-3}|E_T|$]{
    \label{fig:temporal-sx-superuser--error3}
    \includegraphics[width=0.48\linewidth]{out/temporal-sx-superuser-error3.pdf}
  } \\[-2ex]
  \caption{Runtime and Error in ranks obtained with \textit{Static}, \textit{Naive-dynamic (ND)}, \textit{Dynamic Traversal (DT)}, our improved \textit{Dynamic Frontier (DF)}, and our improved \textit{Dynamic Frontier with Pruning (DF-P)} PageRank on the \textit{sx-superuser} dynamic graph. The size of batch updates range from $10^{-5}|E_T|$ to $10^{-3}|E_T|$. The rank error with each approach is measured relative to ranks obtained with a reference Static PageRank run, as detailed in Section \ref{sec:measurement}.}
  \label{fig:temporal-sx-superuser}
\end{figure*}

\begin{figure*}[!hbt]
  \centering
  \subfigure[Runtime on consecutive batch updates of size $10^{-5}|E_T|$]{
    \label{fig:temporal-wiki-talk-temporal--runtime5}
    \includegraphics[width=0.48\linewidth]{out/temporal-wiki-talk-temporal-runtime5.pdf}
  }
  \subfigure[Error in ranks obtained on consecutive batch updates of size $10^{-5}|E_T|$]{
    \label{fig:temporal-wiki-talk-temporal--error5}
    \includegraphics[width=0.48\linewidth]{out/temporal-wiki-talk-temporal-error5.pdf}
  } \\[2ex]
  \subfigure[Runtime on consecutive batch updates of size $10^{-4}|E_T|$]{
    \label{fig:temporal-wiki-talk-temporal--runtime4}
    \includegraphics[width=0.48\linewidth]{out/temporal-wiki-talk-temporal-runtime4.pdf}
  }
  \subfigure[Error in ranks obtained on consecutive batch updates of size $10^{-4}|E_T|$]{
    \label{fig:temporal-wiki-talk-temporal--error4}
    \includegraphics[width=0.48\linewidth]{out/temporal-wiki-talk-temporal-error4.pdf}
  } \\[2ex]
  \subfigure[Runtime on consecutive batch updates of size $10^{-3}|E_T|$]{
    \label{fig:temporal-wiki-talk-temporal--runtime3}
    \includegraphics[width=0.48\linewidth]{out/temporal-wiki-talk-temporal-runtime3.pdf}
  }
  \subfigure[Error in ranks obtained on consecutive batch updates of size $10^{-3}|E_T|$]{
    \label{fig:temporal-wiki-talk-temporal--error3}
    \includegraphics[width=0.48\linewidth]{out/temporal-wiki-talk-temporal-error3.pdf}
  } \\[-2ex]
  \caption{Runtime and Error in ranks obtained with \textit{Static}, \textit{Naive-dynamic (ND)}, \textit{Dynamic Traversal (DT)}, our improved \textit{Dynamic Frontier (DF)}, and our improved \textit{Dynamic Frontier with Pruning (DF-P)} PageRank on the \textit{wiki-talk-temporal} dynamic graph. The size of batch updates range from $10^{-5}|E_T|$ to $10^{-3}|E_T|$. The rank error with each approach is measured relative to ranks obtained with a reference Static PageRank run, as detailed in Section \ref{sec:measurement}.}
  \label{fig:temporal-wiki-talk-temporal}
\end{figure*}

\begin{figure*}[!hbt]
  \centering
  \subfigure[Runtime on consecutive batch updates of size $10^{-5}|E_T|$]{
    \label{fig:temporal-sx-stackoverflow--runtime5}
    \includegraphics[width=0.48\linewidth]{out/temporal-sx-stackoverflow-runtime5.pdf}
  }
  \subfigure[Error in ranks obtained on consecutive batch updates of size $10^{-5}|E_T|$]{
    \label{fig:temporal-sx-stackoverflow--error5}
    \includegraphics[width=0.48\linewidth]{out/temporal-sx-stackoverflow-error5.pdf}
  } \\[2ex]
  \subfigure[Runtime on consecutive batch updates of size $10^{-4}|E_T|$]{
    \label{fig:temporal-sx-stackoverflow--runtime4}
    \includegraphics[width=0.48\linewidth]{out/temporal-sx-stackoverflow-runtime4.pdf}
  }
  \subfigure[Error in ranks obtained on consecutive batch updates of size $10^{-4}|E_T|$]{
    \label{fig:temporal-sx-stackoverflow--error4}
    \includegraphics[width=0.48\linewidth]{out/temporal-sx-stackoverflow-error4.pdf}
  } \\[2ex]
  \subfigure[Runtime on consecutive batch updates of size $10^{-3}|E_T|$]{
    \label{fig:temporal-sx-stackoverflow--runtime3}
    \includegraphics[width=0.48\linewidth]{out/temporal-sx-stackoverflow-runtime3.pdf}
  }
  \subfigure[Error in ranks obtained on consecutive batch updates of size $10^{-3}|E_T|$]{
    \label{fig:temporal-sx-stackoverflow--error3}
    \includegraphics[width=0.48\linewidth]{out/temporal-sx-stackoverflow-error3.pdf}
  } \\[-2ex]
  \caption{Runtime and Error in ranks obtained with \textit{Static}, \textit{Naive-dynamic (ND)}, \textit{Dynamic Traversal (DT)}, our improved \textit{Dynamic Frontier (DF)}, and our improved \textit{Dynamic Frontier with Pruning (DF-P)} PageRank on the \textit{sx-stackoverflow} dynamic graph. The size of batch updates range from $10^{-5}|E_T|$ to $10^{-3}|E_T|$. The rank error with each approach is measured relative to ranks obtained with a reference Static PageRank run, as detailed in Section \ref{sec:measurement}.}
  \label{fig:temporal-sx-stackoverflow}
\end{figure*}

\begin{figure*}[hbtp]
  \centering
  \subfigure[Overall result]{
    \label{fig:8020-runtime--mean}
    \includegraphics[width=0.38\linewidth]{out/8020-runtime-mean.pdf}
  }
  \subfigure[Results on each graph]{
    \label{fig:8020-runtime--all}
    \includegraphics[width=0.58\linewidth]{out/8020-runtime-all.pdf}
  } \\[-1ex]
  \caption{Runtime (logarithmic scale) of \textit{Static}, \textit{Naive-dynamic (ND)}, \textit{Dynamic Traversal (DT)}, our improved \textit{Dynamic Frontier (DF)}, and \textit{Dynamic Frontier with Pruning (DF-P)} PageRank on large (static) graphs with generated random batch updates, on batch updates of size $10^{-7}|E|$ to $0.1|E|$ in multiples of $10$. The updates include $80\%$ edge insertions and $20\%$ edge deletions, simulating realistic changes upon a dynamic graph. The subfigure on the right illustrates the runtime of each approach for each graph in the dataset, while the subfigure of the left presents overall runtimes (using geometric mean for consistent scaling across graphs). In addition, the speedup of each approach, relative to Static PageRank, is labeled on respective lines.}
  \label{fig:8020-runtime}
\end{figure*}

\begin{figure*}[hbtp]
  \centering
  \subfigure[Overall result]{
    \label{fig:8020-error--mean}
    \includegraphics[width=0.38\linewidth]{out/8020-error-mean.pdf}
  }
  \subfigure[Results on each graph]{
    \label{fig:8020-error--all}
    \includegraphics[width=0.58\linewidth]{out/8020-error-all.pdf}
  } \\[-1ex]
  \caption{Error comparison of \textit{Static}, \textit{Naive-dynamic (ND)}, \textit{Dynamic Traversal (DT)}, our improved \textit{Dynamic Frontier (DF)}, and \textit{Dynamic Frontier with Pruning (DF-P)} PageRank on large (static) graphs with generated random batch updates, relative to a Reference Static PageRank (see Section \ref{sec:measurement}), using $L1$-norm. The size of batch updates range from $10^{-7} |E|$ to $0.1 |E|$ in multiples of $10$ (logarithmic scale), consisting of $80\%$ edge insertions and $20\%$ edge deletions to simulate realistic dynamic graph updates. The right subfigure depicts the error for each approach in relation to each graph, while the left subfigure showcases overall errors using geometric mean for consistent scaling across graphs.}
  \label{fig:8020-error}
\end{figure*}


\clearpage

\section{Appendix}

\subsection{Derivation of Closed loop formula for Rank calculation towards Dynamic Frontier with Pruning (DF-P) PageRank}
\label{sec:pr-prune-derivation}

We proceed to derive the closed-loop formula for rank calculation with DF-P PageRank. As outlined in Sections \ref{sec:dataset} and \ref{sec:batch-generation}, self-loops are added to each vertex to circumvent the need for a global teleport rank computation in every iteration, thus reducing overhead. In DF-P PageRank, our aim is to skip the computation of ranks for vertices likely to have already converged. However, the existence of self-loops causes a delay in vertex rank convergence due to the immediate recursive nature they introduce. For instance, if the ranks of all in-neighbors of a vertex have already converged, the presence of self-loops inhibits the convergence of the vertex's rank in a single iteration. Nevertheless, we can mitigate this convergence issue by employing a closed-loop formula for the rank calculation of each vertex.

To achieve this, let us denote $r_0$ as the initial rank of a vertex $v$, $\alpha$ as the damping factor, $c = \sum_{u \in G.in(v)} \frac{R[u]}{|G.out(u)|}$ as the total rank contribution from its in-neighbors (excluding itself), $d = |G.out(v)|$ as its out-degree, and $C_0$ as $1 - \alpha/|V|$. Given the assumption that the rank contribution of its in-neighbors remains constant, the rank of $v$ after one iteration can be expressed as:

\begin{flalign*}
  r_1 & = \alpha (c + \frac{r_0}{d}) + C_0 && \\
      & = \alpha c + \alpha \frac{r_0}{d} + C_0 && \\
\end{flalign*}

\noindent
After the second iteration, the rank of the vertex would be:

\begin{flalign*}
  r_2 & = \alpha (c + \frac{r_1}{d}) + C_0 && \\
      & = \alpha (c + \frac{1}{d} (\alpha c + \alpha \frac{r_0}{d} + C_0)) + C_0 && \\
      & = \alpha c + \alpha^2 \frac{c}{d} + \alpha^2 \frac{r_0}{d^2} + \alpha \frac{C_0}{d} + C_0 &&
\end{flalign*}

\noindent
Following the third iteration, the vertex's rank would be:

\begin{flalign*}
  r_3 & = \alpha (c + \frac{r_2}{d}) + C_0 && \\
      & = \alpha (c + \frac{1}{d} (\alpha c + \alpha^2 \frac{c}{d} + \alpha^2 \frac{r_0}{d^2} + \alpha \frac{C_0}{d} + C_0) + C_0 && \\
      & = \alpha c + \alpha^2 \frac{c}{d} + \alpha^3 \frac{c}{d^2} + \alpha^3 \frac{r_0}{d^3} + \alpha^2 \frac{C_0}{d^2} + \alpha \frac{C_0}{d} + C_0 && \\
\end{flalign*}

\noindent
Expanding this to an infinite number of iterations, the vertex's final rank would be:

\begin{flalign*}
  r_\infty & = \frac{\alpha c}{1 - \alpha / d} + \frac{C_0}{1 - \alpha / d} && \\
           & = \frac{1}{1 - \alpha / d} (\alpha c + C_0)
\end{flalign*}

\noindent
Hence, the closed-loop formula for calculating the rank of a vertex $v$ in DF-P PageRank is:

\begin{flalign}
  R[v] & = \frac{1}{1 - \alpha / |G.out(v)|} \left(\alpha K + \frac{1 - \alpha}{|V|}\right) && \\
    \text{where, } K & = \left(\sum_{u \in G.in(v)} \frac{R[u]}{|G.out(u)|}\right) - \frac{R[v]}{|G.out(v)|}
\end{flalign}

\begin{table}[hbtp]
  \centering
  \caption{List of $12$ graphs sourced from the SuiteSparse Matrix Collection \cite{suite19}, where directed graphs are indicated with $*$. Here, $|V|$ denotes the number of vertices, $|E|$ represents the number of edges (inclusive of self-loops), and $D_{avg}$ represents the average degree.}
  \label{tab:dataset-large}
  \begin{tabular}{|c||c|c|c|c|}
    \toprule
    \textbf{Graph} &
    \textbf{\textbf{$|V|$}} &
    \textbf{\textbf{$|E|$}} &
    \textbf{\textbf{$D_{avg}$}} \\
    \midrule
    \multicolumn{4}{|c|}{\textbf{Web Graphs (LAW)}} \\ \hline
    indochina-2004$^*$ & 7.41M & 199M & 26.8 \\ \hline  % & \num{4.7e-4}
    % uk-2002$^*$ & 18.5M & 311M & 16.8 \\ \hline  % & \num{9.6e-5}
    arabic-2005$^*$ & 22.7M & 654M & 28.8 \\ \hline  % & \num{5.5e-4}
    uk-2005$^*$ & 39.5M & 961M & 24.3 \\ \hline  % & \num{9.6e-5}
    webbase-2001$^*$ & 118M & 1.11B & 9.4 \\ \hline  % & \num{7.3e-7}
    it-2004$^*$ & 41.3M & 1.18B & 28.5 \\ \hline  % & \num{3.8e-4}
    sk-2005$^*$ & 50.6M & 1.98B & 39.1 \\ \hline  % & \num{5.8e-4}
    \multicolumn{4}{|c|}{\textbf{Social Networks (SNAP)}} \\ \hline
    com-LiveJournal & 4.00M & 73.4M & 18.3 \\ \hline  % & \num{7.9e-4}
    com-Orkut & 3.07M & 237M & 77.3 \\ \hline  % & \num{6.7e-2}
    \multicolumn{4}{|c|}{\textbf{Road Networks (DIMACS10)}} \\ \hline
    asia\_osm & 12.0M & 37.4M & 3.1 \\ \hline  % & \num{8.4e-4}
    europe\_osm & 50.9M & 159M & 3.1 \\ \hline  % & \num{6.6e-4}
    \multicolumn{4}{|c|}{\textbf{Protein k-mer Graphs (GenBank)}} \\ \hline
    kmer\_A2a & 171M & 531M & 3.1 \\ \hline  % & \num{9.4e-5}
    kmer\_V1r & 214M & 679M & 3.2 \\ \hline  % & \num{3.2e-4}
  \bottomrule
  \end{tabular}
\end{table}

